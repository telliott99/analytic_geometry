\documentclass[11pt, oneside]{article}   	% use "amsart" instead of "article" for AMSLaTeX format
\usepackage{geometry}                		% See geometry.pdf to learn the layout options. There are lots.
\geometry{letterpaper}                   		% ... or a4paper or a5paper or ... 
\usepackage{graphicx}				% Use pdf, png, jpg, or eps§ with pdflatex; use eps in DVI mode
								% TeX will automatically convert eps --> pdf in pdflatex		
\usepackage{amssymb}
\usepackage{amsmath}
\usepackage{parskip}
\usepackage{color}
\usepackage{hyperref}

\graphicspath{{/Users/telliott/Github/figures/}}
% \begin{center} \includegraphics [scale=0.4] {gauss3.png} \end{center}

%break
\title{Spherical cap}
\date{}

\begin{document}
\maketitle
\Large

%[my-super-duper-separator]


Here is a figure from Wolfram for a spherical cap.  We are interested in formulas for the area and volume of the solid obtained by slicing through a sphere, where the height of the cap that is produced is $h$, and the distance of closest approach to the center of the sphere is $R-h$.
\begin{center} \includegraphics [scale=0.6] {spherical_cap.png} \end{center}

\subsection*{geometry}
If we start from the equator, and think about a thin belt going around the sphere, the belt has length equal to the circumference $2\pi R$ and width $h$, and thus area $S$:
\[ S = 2 \pi R h \]

We believe this should be the formula for the surface area of a belt of width $h$, at least near the equator.  In the figure, this width is labeled as $R-h$, because we are more interested in the cap.  Thus, for the calculation below, this area will be $2\pi R(R-h)$.

Consider that the total surface area of the hemisphere is $2\pi R^2$ so the area of the cap is the difference
\[ S = 2\pi R^2 -  2 \pi R (R-h) = 2 \pi Rh \]

That's a surprising result, that the area of the cap depends only on $R$ and its width (here called $h$).  At least, that is certainly true in the limit as the width of the belt at the equator is very small.

\subsection*{polar cap}
Furthermore, if we look in the figure at the right triangle with $h$ and $a$ as the sides, we can draw the hypotenuse of that triangle and call it $r$ (it's not actually labeled in the figure).  
\begin{center} \includegraphics [scale=0.6] {spherical_cap.png} \end{center}

It is sometimes called the slant height.  We calculate
\[ a^2 = R^2 - (R - h)^2 = 2 Rh - h^2 \]
\[ r^2 = a^2 + h^2 = 2 Rh - h^2 + h^2 = 2 Rh \]

Now think about a very small spherical cap, then it would be almost flat, a circle, and its radius would be $r$ and area
\[ S = \pi r^2 \]
But $r^2 = 2Rh$, so again we have the same formula for the surface area of a small cap and a belt near the equator!

\subsection*{General case}
Consider a belt of width $h$ at a position somewhere in the temperate latitudes of the sphere, not close to either the pole or the equator.

We use a thin belt, so that going north toward the pole the surface of the sphere is approximately flat.  As before, $h$ is the width of the projection of the belt on the $z$-axis.  

The width $w$ of the belt on the surface is larger than $h$, because $w$ is not vertical but tilted toward the $z$-axis.  And since the surface is flat, this angle with the $z$-axis is constant over the width of the belt.

Draw a ray from the center of the sphere to the point where the belt is. The ray makes an angle $s$ with the $xy$-plane, at the center of the sphere.  The radius $a$ at the position of the belt is smaller than $R$ by a factor of $\cos s$. 
\[ a = R \cos s \]
\begin{center} \includegraphics [scale=0.6] {sphcap2.png} \end{center}
The tangent to the sphere at this point (namely, $w$) is perpendicular to the ray.  In the right panel, we see a small triangle on the surface of the sphere with sides $w$ and $h$.

All three of the triangles shown in the right panel are right triangles, with complementary angles $s$ and $t$ (not all of them are labeled).  Can you see that the angle between $h$ and $w$ is $s$?

Therefore, the slant height $w$ of the belt is larger than $h$ by the factor of $\cos s$.
\[ h = w \cos s \]
So the true area is
\[ 2 \pi \ a \ w = 2 \pi R \cos s \frac{h}{\cos s} = 2 \pi R h \]
The cosine of the angle comes in twice, and these factors cancel.  

The formula $2\pi R h$ is correct everywhere.

\end{document}