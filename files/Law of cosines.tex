\documentclass[11pt, oneside]{article} 
\usepackage{geometry}
\geometry{letterpaper} 
\usepackage{graphicx}
	
\usepackage{amssymb}
\usepackage{amsmath}
\usepackage{parskip}
\usepackage{color}
\usepackage{hyperref}

\graphicspath{{/Users/telliott/Dropbox/Github-Math/figures/}}
% \begin{center} \includegraphics [scale=0.4] {gauss3.png} \end{center}

\title{Law of cosines}
\date{}

\begin{document}
\maketitle
\Large

%[my-super-duper-separator]

\label{sec:Law_of_cosines}

\subsection*{Law of cosines}

\begin{center} \includegraphics [scale=0.5] {cosine_law.png} \end{center}

Designate the lengths of a triangle's sides as $a,b,c$ and the angle between sides $a$ and $b$ as $C$ (because it is opposite side $c$).  The law of cosines says that

\[ c^2 = a^2 + b^2 - 2 a b \cos C \]

Lockhart calls this the "generalized" Pythagorean theorem.  We can view the term $-2ab \cos C$ as a correction term which disappears in the case where $\angle C$ is 90 degrees.

If $\angle C$ were a right angle then we would have that
\[ c^2 = a^2 + b^2 \]
but since it's not, there is a correction term
\[ c^2 = a^2 + b^2 - \Delta \]

$\Delta$ is \emph{subtracted} from the length of $c$.  $\Delta$ should be larger, the smaller $C$ gets (as the cosine does), since the opposite side gets squeezed.  And it should be larger as the sides $a$ and $b$ are larger, since there is a bigger effect on $c$ in absolute terms.  

That's exactly what the law of cosines does!  $\Delta$ is some factor times $a$ times $b$ times $\cos C$ and the whole term has a minus sign.

\subsection*{visual proof}
\begin{center} \includegraphics [scale=0.4] {law_of_cosines2.png} \end{center}

Draw a right triangle using one diagonal of a circle (so any third point on the circle forms a right angle), then draw any other diagonal, this time, in red.  

At the point of interest, the red diagonal is divided into $a + c$ and $a - c$.  The total length of the base of the black right triangle (not the hypotenuse) is $2a \cos \theta$ so it is divided into $b$ and $2a \cos \theta - b$.  

We apply the theorem on chords of a circle from (\hyperref[sec:chord_segments]{\textbf{here}}).  The products of the chord segments are equal.

\[ (a + c)(a - c) = (2a \cos \theta - b)(b) \]
\[ a^2 - c^2 = 2ab \cos \theta - b^2 \]
\[ c^2 = a^2 + b^2 - 2ab \cos \theta \]


\subsection*{derivation}
The result also follows from the Pythagorean Theorem.  (In fact, we can reuse the same diagram that was shown for the algebraic proof of the theorem).

For a triangle with sides $a$, $b$ and $c$ and angles opposite those sides $A$, $B$ and $C$, divide the third side into two lengths $c=x+y$ using the vertical altitude from vertex $C$.
\begin{center} \includegraphics [scale=0.4] {triangle6.png} \end{center}

\emph{Proof.}

\[ a^2 = x^2 + h^2 \]
\[ b^2 = y^2 + h^2  \]
So
\[ a^2 - b^2 = x^2 - y^2 \]

Also
\[ y = c - x \]
\[ y^2 = c^2 - 2cx + x^2 \]

So going back to
\[ a^2 - b^2 = x^2 - y^2 \]
\[ = -c^2 + 2cx \]

Rearranging
\[ b^2 = a^2 + c^2 - 2cx \]
Finally, $x = a \cos B$ so
\[ b^2 = a^2 + c^2 - 2ac \cos B \]

This is the law of cosines.

$\square$

Any side of a triangle can be expressed in terms of the other two and the cosine of the angle between them.  Thus, for example
\[ c^2 = a^2 + b^2 - 2ab \cos C  \]
\[ a^2 = b^2 + c^2 - 2bc \cos A  \]

\subsection*{alternate proof}

This approach is a bit longer, but there is a payoff.

\emph{Proof}.

Add the first two equations and rearrange:
\[ a^2 + b^2 = x^2 + y^2 + 2h^2 \]

but 
\[ c^2 = (x + y)^2 = x^2 + y^2 + 2xy \]

so
\[ a^2 + b^2 = c^2 - 2xy + 2h^2 \]
\[ c^2 = a^2 + b^2 + 2(xy - h^2) \]
\[ = a^2 + b^2 - 2(h^2 - xy) \]

We need to show that $h^2 - xy$ somehow equals $ab \cos C$.  

\begin{center} \includegraphics [scale=0.4] {triangle6.png} \end{center}

Let the smaller right triangle with hypotenuse $a$ include angle $C'$, and the one with hypotenuse $b$ have angle $C''$, where $C = C' + C''$.

\[ h = a \cos C' \]
\[ x = a \sin C' \]
\[ h = b \cos C'' \]
\[ y = b \sin C' \]

This reminds us of the sum of cosines:
\[ \cos C = \cos C' \cos C'' - \sin C' \sin C'' \]

Let's see:

\[ h^2 = ab \cos C' \cos C'' \]
\[ xy = ab \sin C' \sin C'' \]

So
\[ h^2 - xy = ab(\cos C' \cos C'' - \sin C' \sin C'') \]
\[ = ab \cos C \]

Substituting into the equation we had above:

\[ c^2 = a^2 + b^2 - 2(h^2 - xy) \]
\[ c^2 = a^2 + b^2 - 2ab \cos C \]

$\square$

We can view the second as a proof of the formula for sum of cosines, in reverse.

\subsection*{other triangles}

It is natural when proving theorems about triangles to draw an acute triangle as the example.  Often, a proof must be adjusted for the case of obtuse triangles.  We won't usually show these extensions, but here is a proof of the law of cosines for an obtuse triangle.

\begin{center} \includegraphics [scale=0.4] {law_of_cosines_obtuse.png} \end{center}

\emph{Proof}.

We draw two right triangles by extending the side adjacent to the obtuse angle.  Let $x$ be the length of $AD$ and $h$ be the length of $BD$.  By the Pythagorean theorem:

\[ c^2 = x^2 + h^2 \]
\[ a^2 = (b + x)^2 + h^2 \]

By subtraction
\[ a^2 - c^2 = (b+x)^2 - x^2 \]
\[ = b^2 + 2bx \]
So
\[ a^2 = b^2 + c^2 + 2bx \]

It remains to show that $x = -c \cos A$.  

But $x = c \cos \angle BAD$, and for two supplementary angles, the cosine of one is minus the cosine of the other.

$\square$

\subsection*{Law of sines}
Here's a simple identity called the law of sines.  In contrast to the law of cosines, it is fairly trivial to prove.
\begin{center} \includegraphics [scale=0.4] {triangle4.png} \end{center}

\[ \frac{h}{b} = \sin A \]
\[ \frac{h}{a} = \sin B \]
Therefore
\[ h = b \sin A = a \sin B \]
\[ \frac{\sin A}{a} = \frac{\sin B}{b} \]
We could do the same construction and argument with either $A$ or $B$, and the third angle, call it $C$.  Therefore
\[ \frac{\sin A}{a} = \frac{\sin B}{b} = \frac{\sin C}{c} \]

The sine of an angle, divided by the length of the side opposite, is a constant.

\subsection*{algebra}

I came across the following two problems on the web

\url{https://brownmath.com/twt/intro.htm}

To do:  "solve" a triangle (find all three sides) from the following information:

$\circ$ \ \ the area plus two angles

or

$\circ$ \ \ the area plus two sides

\begin{center} \includegraphics [scale=0.4] {triangle5.png} \end{center}

For the first one, if we know two angles, we know the third.  Let the area be $K$.  For any base side the altitude is the adjacent side times the sine of the angle.  So, for example

\[ K = \frac{1}{2} \ b \sin A \cdot c \]
or 
\[ K = \frac{1}{2} ab \sin C \]

We must find $b$ in terms of $a$, for example, as $b = a \sin B /\sin A$ so substituting:
\[ K = \frac{1}{2} a \cdot a \frac{\sin B}{\sin A} \ \sin C \]
so then 
\[ a^2 = 2K \frac{\sin A}{\sin B \sin C} \]
and then we get the other two sides from the law of sines.

For the second problem, area plus two sides, we could perhaps start with Heron's formula and obtain a fourth-order polynomial in $c$.

Perhaps an easier approach is as follows.  Suppose we're given $a$ and $b$ so (as before)

\[ K = \frac{1}{2} ab \sin C \]

From which we obtain $\cos C$
\[ \cos C = \sqrt{1 - (\frac{2K}{ab})^2 } \]

and then use the law of cosines:

\[ c^2 = a^2 + b^2 - 2ab \cos C \]
\[ c^2 = a^2 + b^2 - 2ab \sqrt{1 - (\frac{2K}{ab})^2} \]

which can be simplified as

\[ c^2 = a^2 + b^2 -  \sqrt{4a^2b^2 - 16K^2} \]

And that factor of $16K^2$ makes me think this is probably related to the derivation of Heron's law, but let's wait and take a look.

\end{document}