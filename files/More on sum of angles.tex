\documentclass[11pt, oneside]{article} 
\usepackage{geometry}
\geometry{letterpaper} 
\usepackage{graphicx}
	
\usepackage{amssymb}
\usepackage{amsmath}
\usepackage{parskip}
\usepackage{color}
\usepackage{hyperref}

\graphicspath{{/Users/telliott/Dropbox/Github-Math/figures/}}
% \begin{center} \includegraphics [scale=0.4] {gauss3.png} \end{center}

\title{Sum of angles}
\date{}

\begin{document}
\maketitle
\Large

%[my-super-duper-separator]

Here are several other derivations that I've come across over the years.  You only need what we've given above, but they are interesting to work through and give practice in dealing with trig functions.

\subsection*{calculus}

All you need to know for this is that the derivative of sine is cosine and the derivative of the cosine is minus the sine.  If you don't know about derivatives yet, just skip this example.

Suppose we know one of the formulas already, say
\[ \cos s + t = \cos s \cos t - \sin s \sin t \]

Treat $t$ as a constant and take the derivative with respect to $s$.  The left-hand side is

\[ \frac{d}{ds} \ \cos s + t = - \sin s + t \]
whlle the right-hand side is:
\[ \frac{d}{ds} \ ( \cos s \cos t - \sin s \sin t) = - \sin s \cos t - \cos s \sin t \]

But these are equal!  Multiplying by $-1$, we have that
\[ \sin s + t = \sin s \cos t + \cos s \sin t \]

Going the other way works as well, as does treating $t$ as the variable.

\subsection*{vector rotation}

I want to take a moment to introduce a concept that gives another simple proof of the sum of angles theorems.  We don't have time to cover it in the detail it really deserves, that's for later.

Start with the idea of a \emph{vector} from the origin to a point.  Vectors have length --- the vectors depicted are unit vectors with length $1$.  They also have a direction.  The unit vector along the $x$-axis can be represented as the pair of $x,y$-coordinates $(1,0)$.

Let us rotate the vector in the counter-clockwise direction by an angle $\phi$.  The new coordinates of the head of the vector (marked with the arrow) are $(\cos \phi, \sin \phi)$.  We have
\[ 1, 0 \rightarrow \cos \phi, \sin \phi \]
for a vector $x$ units long we have
\[ x, 0 \rightarrow x \cos \phi, x \sin \phi \]
 
By similar reasoning
\[ 0, 1 \rightarrow - \sin \phi, \cos \phi \]
\[ 0, y \rightarrow - y \sin \phi, y \cos \phi \]
We get a minus sign because the rotated unit vector that started pointing straight up is now in the second quadrant, the $x$-component is negative.

\begin{center} \includegraphics [scale=0.4] {rotate_vectors.png} \end{center}

For a general vector $(x,y)$ the resulting vector is $(x',y')$ and the components are:
\[ x' = x \cos \phi - y \sin \phi \]
\[ y' = x \sin \phi + y \cos \phi \]

Now, suppose we rotate a second time, by an angle $\theta$.  Just add a prime for each and substitute $\theta$ for $\phi$.
\[ x'' = x' \cos \theta - y' \sin \theta \]
\[ y'' = x' \sin \theta + y' \cos \theta \]

Write $x''$ in terms of the original $x$ and $y$:
\[ x'' = x' \cos \theta - y' \sin \theta \]
\[ = (x \cos \phi - y \sin \phi) \cos \theta - (x \sin \phi + y \cos \phi) \sin \theta \]
\[ = x \cos \phi \cos \theta - y \sin \phi \cos \theta - x \sin \phi \sin \theta - y \cos \phi \sin \theta \]
\[ = x \ [ \ \cos \phi \cos \theta - \sin \phi \sin \theta \ ] \ - y \ [ \ \sin \phi \cos \theta + \cos \phi \sin \theta \ ] \]

The key idea is that we must get the same result if we turn through both angles at once:
\[ x'' = x ( \cos \phi + \theta) - y (\sin \phi + \theta) \]

The cofactors of $x$ and $y$ must separately be equal:
\[ \cos \phi + \theta = \cos \phi \cos \theta - \sin \phi \sin \theta \]
\[ \sin \phi + \theta = \sin \phi \cos \theta + \cos \phi \sin \theta \]

Those are our formulas!

If you know about matrix multiplication, you can write 

\[ 
\begin{bmatrix}
\ \ \cos s \ \sin s \\
-\sin s \ \cos s 
\end{bmatrix}
\]
and then
\[
\begin{bmatrix}
\ \ \cos s \ \sin s \\
-\sin s \ \cos s 
\end{bmatrix}
\begin{bmatrix}
\ \ \cos t \ \sin t \\
-\sin t \ \cos t 
\end{bmatrix}
= \begin{bmatrix}
\ \ \cos s+t \  \sin s+t \\
- \sin s+t \ \ \cos s+t
\end{bmatrix}
\]
The simple multiplication rule will recover our formulas.

\subsection*{Shankar}

In his wonderful books on Physics, Shankar sets things up a little differently, so let's just take a quick look at his diagram.

\begin{center} \includegraphics [scale=0.6] {Shankar2_4b.png} \end{center}

The angle of rotation of the coordinate system going from $x,y$ to $x',y'$ (counter-clockwise from the x-axis) is $\phi$, labeled with a black dot.  The rotated $y$-axis, called $y'$, makes the same angle with the original $y$-axis, as shown at the left.

The angle between $\mathbf{A}$ and the x-axis is $\phi + \theta$.  Since $A_x'$ is the length of the projection of the vector on the $x'$-axis, the dotted line near the label $A_y'$ is perpendicular to the $x'$ -axis.  (The diagram is not perfectly square).

Therefore, there are two angles each equal to the sum of the red and block dots on both ends of the diagram, by the alternate interior angles theorem.

The angles labeled $\theta$ are equal because they are complementary with red plus black.  Alternatively, use alternate interior angles with the $y$-axis and the dotted line perpendicular to the $x$-axis.

Now

\[ A_x = A \cos \ (\phi + \theta) \]
\[ = A \ [ \ \cos \phi \cos \theta - \sin \phi \sin \theta \ ] \]

\[ A_x' = A \cos \theta \]
\[ A_y' = A \sin \theta \]

\begin{center} \includegraphics [scale=0.6] {Shankar2_4b.png} \end{center}

so
\[ A_x = A_x' \cos \phi - A_y' \sin \phi \]

In a similar way
\[ A_y = A \ \sin (\phi + \theta) \]
\[ = A \ [ \ \sin \phi \cos \theta + \cos \phi \sin \theta \ ] \]

so 
\[ A_y = A_x' \sin \phi + A_y' \cos \phi \]

We rotated the coordinate system counter-clockwise, which would normally give a clockwise-rotation of the vectors, and would result in the minus sign being in the formula for the $y$-component.  

However, the calculation was set up to find $A_x$ in terms of $A_x'$ and $A_y'$.  So the primed components (after the rotation) were really used as the starting point.  

This explains why we have a minus sign in the formula for $A_x$ above (i.e. - $A_y' \sin \theta$), rather than in the formula for $A_y$.

To go the other way, start with the two equations:

\[ A_x = A_x' \cos \phi - A_y' \sin \phi \]
\[ A_y = A_x' \sin \phi + A_y' \cos \phi \]

Multiply the first by $\cos \phi$ and the second by $\sin \phi$ and add (using $\sin^2 \phi + \cos^2 \phi = 1$):
\[ A_x \cos \phi + A_y \sin \phi = A_x' \]

Also multiply the first by $\sin \phi$ and the second by $\cos \phi$ and subtract the first from the second
\[ A_y \cos \phi - A_x \sin \phi = A_y' \]

The minus sign has switched to the second equation.

Finally, we must have that the length of the vector $\mathbf{A}$ is the same in either coordinate system.  The length squared is

\[ A_x^2 = (A_x')^2 \cos^2 \phi + (A_y')^2 \sin^2 \phi - 2 A_x' A_y' \cos \phi \sin \phi \]
\[ A_y^2 = (A_x')^2 \sin^2 \phi + (A_y')^2 \cos^2 \phi + 2 A_x' A_y' \cos \phi \sin \phi \]

Add them together and use our favorite identity to obtain:
\[ A_x^2 + A_y^2 = (A_x')^2 +  (A_y')^2 \]

\end{document}