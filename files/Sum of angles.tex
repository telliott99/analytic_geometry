\documentclass[11pt, oneside]{article} 
\usepackage{geometry}
\geometry{letterpaper} 
\usepackage{graphicx}
	
\usepackage{amssymb}
\usepackage{amsmath}
\usepackage{parskip}
\usepackage{color}
\usepackage{hyperref}

\graphicspath{{/Users/telliott/Dropbox/Github-Math/figures/}}
% \begin{center} \includegraphics [scale=0.4] {gauss3.png} \end{center}

\title{Sum of angles}
\date{}

\begin{document}
\maketitle
\Large

%[my-super-duper-separator]

\label{sec:sum_angles_similar_tri}

There are some really important formulas that relate the sine and cosine of individual angles to the sine and cosine of their sum (or difference).  Here is one of them.  For angles $s$ and $t$

\[ \cos s - t = \cos s \cos t + \sin s \sin t \]

By $\cos s - t$ we mean $\cos (s - t)$, but have left off the parentheses. 

There are four formulas, and then some special examples.  These are used a lot in calculus, not only for solving problems, but most important, in finding an expression for the derivatives of the sine and cosine functions.

You really must know them.  I think it's so important that we will show several ways of finding them.  

The proofs are also beautiful, which helps to explain why I've included so many.  The easiest way to remember them uses Euler's equation, but since you've probably never seen that, I'll put it off until the end.

I've memorized only the one given above.  Say "cos cos" and then recall the difference in sign, minus on the left, plus on the right.

I like this version because it can be checked easily.  Just set $s = t$:
\[ \cos s - t = \cos s - s = \cos 0 = 1 = \cos^2 s + \sin^2 s \]
which is our favorite trigonometric identity and obviously correct.

\subsection*{similar triangles}

Draw two triangles, one on top of the other, with the hypotenuse of the second scaled to be equal to $1$.  Then draw a rectangle around the whole thing.

\begin{center} \includegraphics [scale=0.4] {sum_angles_6.png} \end{center}

For the triangle with angle $\theta$ and hypotenuse $1$, the labels should be obvious.

Second, for triangles with angle $\phi$ where the hypotenuse is \emph{not} $1$, we have something like $\cos \phi \cos \theta$ on the bottom of the figure, which gives the desired value $\cos \phi$ after dividing by the hypotenuse, $\cos \theta$.

The angle labeled $\theta + \phi$ at the top is known by the alternate interior angles theorem, and the angle $\phi$ at top right is by complementary and supplementary angles.

Now, just read off the relationships from the sides of the rectangle:
\[ \sin \phi + \theta = \sin \phi \cos \theta + \cos \phi \sin \theta \]
\[ \cos \phi + \theta = \cos \phi \cos \theta - \sin \phi \sin \theta \]

\subsection*{change signs}

For $\cos s - t$, flip the sign on the second term.  
\[ \cos s - t = \cos s \cos t + \sin s \sin t \]
That's because
\[ \cos -\theta = \cos \theta \]
\[ \sin - \theta = - \sin \theta \]

\begin{center} \includegraphics [scale=0.4] {pm_theta.png} \end{center}

The diagram shows the reason:
\[ \cos \theta = x/r = \cos - \theta \]
while
\[ \sin \theta = y/r = -  (\sin - \theta ) = - (-y/r) \]

Substitute $- \sin \theta$ for $\sin - \theta$ and $\cos \theta$ for $\cos - \theta$:
\[ \cos s - t = \cos s \cos - t + \sin s \sin - t \]
\[ = \cos s \cos t - \sin s \sin t \]
and
\[ \sin s - t = \sin s \cos t - \cos s \sin t \]

It's kind of overkill, but still worth noting that a simple change to the figure we had above will give the difference formulas:

\begin{center} \includegraphics [scale=0.4] {sum_angles_7.png} \end{center}
We've changed the symbol $\phi$ to refer to the complementary angle from what it was before.  

We can justify the label $\phi - \theta$ for the angle at the lower left in various ways, for example, by adding up the three angles at that corner:
\[ (\phi - \theta) + \theta + (90 - \phi) = 90 \]

Switch the labels appropriately (it's easy since this $\phi$ is the complement of the old one).

Read the result:
\[ \sin \phi - \theta = \sin \phi \cos \theta - \cos \phi \sin \theta \]
\[ \cos \phi - \theta = \cos \phi \cos \theta + \sin \phi \sin \theta \]


Here are several other derivations that I've come across over the years.  You only need what we've given above, but they are interesting to work through and give practice in dealing with trig functions.

\subsection*{Strang}

For a geometric derivation of the sum of angles formula with minimal setup, I really like this figure from Strang

\begin{center} \includegraphics [scale=0.6] {strang_sum.png} \end{center}

We have the same triangle in the two panels, just rotated clockwise on the right.

The squared distance between two points in the plane is
\[ d^2 = (x_2 - x_1)^2 + (y_2 - y_1)^2 = \Delta x^2 + \Delta y^2 \]
This is just the Pythagorean theorem in disguise.

In the left panel, $t$ is the angle between the lower radius and the $x$-axis, $s$ is the angle between the upper radius and the $x$-axis, and as labeled, $s-t$ is the angle between the two radii.

The distance $d$ squared for the two points on the circle in the left panel is
\[ d^2 = (\cos s - \cos t)^2 + (\sin s - \sin t)^2 \]

Multiply out:
\[ d^2 = \cos^2 s - 2 \cos s \cos t  + \cos^2 t +  \sin^2 s - 2 \sin s \sin t + \sin^2 t \]
We have two copies of $\sin^2 + \cos^2$, one for angle $s$  and one for angle $t$
\[ d^2 = 2 - 2 \cos s \cos t - 2 \sin s \sin t \]

In the right panel, the two radii have been rotated, preserving the same angle between them.
\[  d^2 = (\cos (s-t) - 1)^2 + \sin(s-t)^2 \]
(Don't forget the $1$).
\[ = \cos^2 (s-t) - 2 \cos(s-t) + 1 + \sin^2 (s-t) \]
\[ = 2 - 2 \cos(s-t) \]

Because the included angle hasn't changed, neither has the distance, so we can equate the two expressions.  
\[ 2 - 2 \cos(s-t) = 2 - 2 \cos s \cos t - 2 \sin s \sin t \]
Subtract 2 from both sides, divide by $2$, and change all the signs leaving
\[ \cos (s - t) = \cos s \cos t + \sin s \sin t \]
This is our formula for the cosine of the difference of two angles.

\subsection*{getting to sine}

Strang's derivation gives us the formula for sum and difference of cosines.  To get the formula for the sine in the same way, we would need to mix sine and cosine when computing a distance in his diagram.  I don't know how to do that.  So our problem is how to go from the cosine formula to the sine formula.

Let's look at the relationships between sine and cosine for angles that are related by addition or subtraction of $\pi/2$.
\begin{center} \includegraphics [scale=0.4] {angles2.png} \end{center}

In the figure, I have simply rotated the same triangle.

What we see is that 
\[ \sin \theta + \frac{\pi}{2} = b = \cos \theta \]
\[ \cos \theta + \frac{\pi}{2} = -a = - \sin \theta \]
and
\[ \sin \theta - \frac{\pi}{2} = -b = - \cos \theta \]
\[ \cos \theta - \frac{\pi}{2} = a = \sin \theta \]

Here is an alternative connection which proceeds from the graph of sine and cosine versus the angle.

\begin{center} \includegraphics [scale=0.4] {sine_cosine_wikipedia.png} \end{center}

so it's easy to see that $\cos t = \sin t + \pi/2$.

So then, consider

\[ \cos s + t = \cos s \cos t - \sin s \sin t \]

Suppose we modify the left-hand side to be
\[ \cos s + t - \frac{\pi}{2} \]
that gives 
\[ \cos (s + t - \frac{\pi}{2}) = \cos s \cos (t - \frac{\pi}{2}) - \sin s \sin (t - \frac{\pi}{2}) \]

If you go back to our table above and substitute for $\cos (t - \frac{\pi}{2}) = \sin t$
\[ \cos (s + t - \frac{\pi}{2}) = \cos s \sin t - \sin s \sin (t - \frac{\pi}{2}) \]
then for $\sin (t - \frac{\pi}{2}) = - \cos t$
\[ \cos (s + t - \frac{\pi}{2}) = \cos s \sin t + \sin s \cos t \]
and then finally for $\cos (s + t - \frac{\pi}{2}) = \sin s + t$ 
\[ \sin s + t = \cos s \sin t + \sin s \cos t \]

Most people find this difficult to do without making a mistake.

\subsection*{sine of the sum}

Here is a very nice geometrical derivation for the sine of the sum.  If you like Strang's derivation for the cosine (as I do), this would be a good complement.

We compute the area of the triangle in two different ways.  On the left we have that
\[ A = \frac{1}{2} ab \ \sin (\theta + \phi) \]

\begin{center} \includegraphics [scale=0.4] {sum_angles_8.png} \end{center}

On the right, the two smaller triangles are right triangles with $h = a \cos \phi = b \cos \theta$.
\[ A_{top} = \frac{1}{2} a \sin \phi \ b \cos \theta \]
\[ A_{bottom} = \frac{1}{2} b \sin \theta \ a \cos \phi \]
These two expressions add to give the total area.  We can factor out the common $ab/2$ and write the equality:
\[ A = \sin (\theta + \phi) = \sin \phi \cos \theta + \sin \theta \cos \phi \]

\url{https://www.cut-the-knot.org/triangle/SinCosFormula.shtml}

\subsection*{Euler}
Euler's formula is:
\[ e^{i \theta} = \cos \theta + i \sin \theta \]

If you've never seen it before, don't worry what it means or where it comes from.  Just treat $i$ as a constant with $i^2 = -1$.  Multiply as follows:
\[ (\cos s + i \sin s)(\cos t + i \sin t) \]
\[ = \cos s \cos t + i^2 \sin s \sin t + i \ [ \ \sin s \cos t + \cos s \sin t \ ] \] 
\[ = \cos s \cos t - \sin s \sin t + i \ [ \ \sin s \cos t + \cos s \sin t \ ] \] 

This is a \emph{complex} number with a real part (the first two terms), plus an imaginary part, the last two terms, with a leading factor of $i$.

For the same calculation with the exponential
\[ e^{is} \cdot e^{it} = e^{i(s+t)} \]
\[ = \cos (s + t) + i \sin (s + t) \]

By Euler's formula, these two expressions are equal.  

The rule for equality of complex numbers is that both the real parts and the imaginary parts must be equal.  So we have
\[ \cos (s + t) = \cos s \cos t - \sin s \sin t \]
\[ \sin (s + t) = \sin s \cos t + \cos s \sin t \]

\subsection*{"a tricky problem"}

Acheson presents this problem (though he does not show the solution).  The algebra defeated me for quite a long time, but I finally got it.

We are given only the radii of the small circles be $a,b$ and $c$, and are asked to find $r$.

\begin{center} \includegraphics [scale=0.4] {Acheson_G163.png} \end{center}

\emph{Solution}.

I have used $R$ for the radius of the large circle.

Let the relevant half-angles be $\alpha, \beta$, and $\gamma$.  These are half angles, so they sum to be one right angle.  Therefore, any one angle is complementary to the sum of the other two.  

\[ \sin \alpha = \cos \beta + \gamma = \cos \beta \cos \gamma - \sin \beta \sin \gamma \]

Our previous work with this problem (in the geometry book)
\begin{center} \includegraphics [scale=0.5] {double_scoop1.png} \end{center}

showed that 
\[ \sin \theta = \frac{R - r}{R + r}, \ \ \ \ \ \ \cos \theta = \frac{2\sqrt{Rr}}{R + r} \]

So the expression above can be written as 
\[ \frac{R - a}{R + a} = \frac{2\sqrt{Rb}}{R + b} \cdot \frac{2\sqrt{Rc}}{R + c} - \frac{R - b}{R + b} \cdot \frac{R - c}{R + c} \]

Rearranging:
\[ (R-a)(R+b)(R+c) + (R+a)(R-b)(R-c) = 4R \sqrt{bc} \cdot (R + a) \]

Obviously, the left-hand side is a cubic in $R$, but the $R^0$ terms on the left cancel, and  there is a factor of $R$ on the right to reduce the equation to a quadratic.

The first term is
\[ (R-a)(R+b)(R+c) \]
\[ = (R-a)(R^2 + Rb + Rc + bc) \]
\[ = R^3 + R^2b + R^2c + Rbc -R^2a - Rab - Rac - abc \]

The second term is 
\[ (R+a)(R-b)(R-c) \]
\[ = (R+a)(R^2 - Rb - Rc + bc) \]
\[ = R^3 - R^2b - R^2c + Rbc + R^2a - Rab - Rac + abc \]

The sum has four cancelations and four terms remaining:
\[ 2R^3 + 2Rbc - 2Rab - 2Rac =  4R \sqrt{bc} \cdot (R + a) \]

We can factor out $2R$
\[ R^2 + bc - ab - ac =  2 (R + a) \sqrt{bc} \]
\[ R^2 + bc - ab - ac =  2R \ \sqrt{bc} + 2a \sqrt{bc} \]

This is a quadratic in $R$:
\[ R^2 - 2 \sqrt{bc} \ R + bc - ab - ac - 2a \sqrt{bc} \]

The discriminant is
\[ 4bc - 4(bc - ab - ac - 2a \sqrt{bc}) \]

with another cancelation
\[ 4ab + 4ac + 8a \sqrt{bc} \]

The $4$ comes out from the square root as $2$ and we have
\[ R = \frac{2 \sqrt{bc} \pm 2 \sqrt{ab + ac + 2a \sqrt{bc}}}{2} \]
\[ R = \sqrt{bc} \pm \sqrt{ab + ac + 2a \sqrt{bc}} \]

But what is under the square root is a perfect square, namely
\[ ab + ac + 2a \sqrt{bc} = (\sqrt{ab} + \sqrt{ac})^2 \]

We have
\[ R = \sqrt{bc} \pm (\sqrt{ab} + \sqrt{ac}) \]

We will disregard the negative sign and obtain finally
\[ R = \sqrt{bc} + \sqrt{ab} + \sqrt{ac} \]

which is the answer we were given.  As expected, it is symmetric in $a,b$ and $c$.

$\square$


\end{document}