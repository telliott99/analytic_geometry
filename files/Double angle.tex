\documentclass[11pt, oneside]{article} 
\usepackage{geometry}
\geometry{letterpaper} 
\usepackage{graphicx}
	
\usepackage{amssymb}
\usepackage{amsmath}
\usepackage{parskip}
\usepackage{color}
\usepackage{hyperref}

\graphicspath{{/Users/telliott/Dropbox/Github-Math/figures/}}
% \begin{center} \includegraphics [scale=0.4] {gauss3.png} \end{center}

\title{Double angle formulas}
\date{}

\begin{document}
\maketitle
\Large

%[my-super-duper-separator]

\label{sec:double_angle}

To review quickly, sine:
\[ \sin s + t = \sin s \cos t + \cos s \sin t \]
\[ \sin 2t = 2 \sin t \cos t \]

And cosine:
\[ \cos s + t = \cos s \cos t - \sin s \sin t \]
\[ \cos 2t = \cos^2 t - \sin^2 t \]
\[ = 2 \cos^2 t - 1 \]

Sometimes it is helpful to have a simpler notation, especially when we want to do algebra.  Let's use the symbols $S,C,$ and $T$ for sine, cosine and tangent, respectively.

Mark the values we are deriving (for one-half a double angle) with primes.  We have:
\[ S = 2S'C' \]
\[ C = C'^2 - S'^2 = 2C'^2 - 1 \]

The inverse double tangent is then

\[ \frac{1}{T} = \frac{2C'^2 - 1}{2S'C'} \]
\[ = \frac{1}{T'} - \frac{1}{S} \]
\[ \frac{1}{T'} = \frac{1}{T} + \frac{1}{S}  \]

This is the formula used to such great effect by Archimedes in obtaining an approximation for $\pi$.

\[ \cot t = \cot 2t + \csc 2t \]

Another formula that is often included in tables is one for the double tangent.  We derive this by going back to the original addition formulas:

\[ \tan s + t = \frac{\sin s + t}{\cos s + t} \]
\[ = \frac{\sin s \cos t + \cos s \sin t}{\cos s \cos t - \sin s \sin t} \]
Divide through by $\cos s \cos t$:
\[ = \frac{\tan s + \tan t}{1 - \tan s \tan t} \]

For the double-angle, this becomes:
\[ \tan 2t = \frac{2 \tan t}{1 - \tan^2 t} \]

\subsection*{David Bowie problem and the half-angle formulas}

Here's a problem from the web.  Two chords are drawn in an octagon, and we're asked what fraction of the area is contained between them.

\begin{center} \includegraphics [scale=0.4] {bowie1.png} \end{center}

\subsection*{inscribed in a square}

The simplest way to look at this problem is to imagine the octagon inscribed in a square.  Tilt the square so that the bottom edge of the octagon is horizontal.

\begin{center} \includegraphics [scale=0.4] {bowie0.png} \end{center}

We can calculate the total area of the octagon as the area of the square minus the small triangles at the corners.  Focusing on them for a minute, we see that each is an isosceles right triangle with sides of length $a$ and hypotenuse $s$.
\[ a^2 + a^2 = 2a^2 = s^2 \]

The area of each triangle is $a^2/2$ and there are four of them for $2a^2$.  

Each side of the square has $s$ plus two copies of $a$, so $b = s + 2a$ and the total area is $b^2$ or
\[ A = (s + 2a)^2 = s^2 + 4as + 4a^2 \]
Subtract $2a^2$
\[ A_{\text{ octagon}} = s^2 + 4as + 2a^2 \]

This form obscures an important result.  Substitute $s^2 = 2a^2$ from above:
\[ A_{\text{ octagon}} = 2a^2 + 4as + 2a^2 = 4(a^2 + as) \]

We have three lines that divide the octagon into four regions.  The total area is four times something.  Is this a coincidence?

The area of two triangles is 
\[ sb = s(s + 2a) = s^2 + 2as = 2(a^2 + as) \]

The triangles are one-half the total area of the octagon.  The two trapezoids must be the other half.

The area of one trapezoid is (subtracting two triangles):
\[ A_{\text{ trapezoid}} = ab - a^2 \]
\[ = a(s + 2a) - a^2 = as + a^2 \]

So two triangles are equal in area to two trapezoids.

\subsection*{inscribed in a circle}

Another way to look at this problem is to imagine the octagon inscribed in a circle.  All the points of the octagon lie on the circle, which is of radius $r$, or diameter $d = 2r$ (below, left panel).

We reason that the two triangles are congruent (by symmetry) and they are both right triangles with a shared hypotenuse that is a diameter of the circle.  

\begin{center} \includegraphics [scale=0.4] {bowie1b.png} \end{center}

How do we know they are right triangles?

The external angle of the octagon is $e = \pi/4$ (we must turn by a full $2 \pi$ in 8 turns so $8e = 2\pi$).

Two successive turns forms one right angle.  Since $\theta + e = \pi/2$ and $e = \pi/4$, we have that $e = \theta = \pi/4$.  The supplementary angle to $e + \theta $ is also a right angle.

Back to the triangles.  The whole angle subtended is $2 \phi$ (by symmetry) and since $2 \phi$ is a peripheral angle of the circle, its arc is $1/4$ of the total, or $\pi/2$. 

As a peripheral angle the measure of $2 \phi$ is one-half that.  Thus, $2 \phi = \pi/4$ and $\phi = \pi/8$.

\begin{center} \includegraphics [scale=0.4] {bowie2.png} \end{center}

\subsection*{trigonometry}

The sides of the triangle can be found in terms of the diameter as $d \sin \phi$ and $d \cos \phi$.  The area is

\[ A_{\text{ triangle}} = \frac{1}{2} \ d \sin \phi \cdot d \cos \phi \]

and the whole area for the region in red (two triangles) is
\[ A_{\text{ region}} = d^2 \sin \phi \cos \phi \]

The question asked what fraction of the octagon's area is shaded red.  

In the geometry book we derived a simple formula for the area of a regular polygon:
\[ A_{\text{ polygon}} = nr^2 \sin \frac{\pi}{n} \cos  \frac{\pi}{n} \]

This is an octagon so $n = 8$.  The angle is $\phi = \pi/8$ and since $r = d/2$

\[ A_{\text{ octagon}} = 8 \cdot \frac{d^2}{2^2} \sin  \phi \cos  \phi \]
\[ = 2 \cdot d^2 \sin  \phi \cos \phi \]

We have again that the area of the entire octagon is twice that of the two triangles.  The octagon is divided by three lines into four equal parts.

\begin{center} \includegraphics [scale=0.4] {bowie2.png} \end{center}

\subsection*{calculating the area}

If we want to actually calculate the area of the octagon, going back to the formula

\[ A_{\text{ octagon}} = 2 \cdot d^2 \sin \frac{\pi}{8} \cos  \frac{\pi}{8} \]

The individual terms with sine and cosine of $\pi/8$ will take some figuring, but we know from the double angle formula that

\[ \sin 2s = 2 \sin s \cos s \]

The sine of $\pi/4$ is $1/\sqrt{2}$, so
\[ \sin \phi \cos \phi = \frac{1}{2 \sqrt{2}} \]

and the area is then 
\[ A_{\text{ octagon}} = \frac{2d^2}{2 \sqrt{2}} = \frac{d^2}{\sqrt{2}} \]

Previously, we calculated the area in terms of the length of the side.
\[ A_{\text{ octagon}} = 4(a^2 + as) = 4(\frac{s^2}{2} + \frac{s^2}{\sqrt{2}}) \]
\[ = 2s^2(1 + \sqrt{2}) \]

If these are truly equal then the relationship between $d$ and $s$ is
\[  \frac{d^2}{\sqrt{2}} = 2s^2(1 + \sqrt{2}) \]
\[ d^2 = s^2(2 \sqrt{2} + 4) \]

\begin{center} \includegraphics [scale=0.4] {bowie2.png} \end{center}

For one triangle, from Pythagoras we have that 
\[ d^2 = s^2 + (s + 2a)^2 \]
\[ = s^2 + s^2 + 4as + 4a^2 \]
with $a = s/\sqrt{2}$ and $2a^2 = s^2$
\[ d^2 = 2s^2 + 2\sqrt{2} s^2 + 2s^2 \]
\[ = s^2 (4 + 2 \sqrt{2}) \]

which matches what we had above.

As another check, write
\[ \sin \frac{\pi}{8} = \frac{s}{d} = \frac{1}{\sqrt{(4 + 2 \sqrt{2})}} = \frac{1}{2 \sqrt{(1 + 1/\sqrt{2})}} \]

I obtained $0.382683..$ for both sides, using a calculator.

\subsection*{half-angle formulas}

Another way to find the relationship between $s$ and $d$ is to use the half-angle formulas.  We take the double angle formula for cosine and rewrite it as follows
\[ \cos 2s = \cos^2 s - \sin^2 s \]
\[ = 2 \cos^2 s - 1 \]

So
\[ \cos^2 s = \frac{1 + \cos 2s}{2} \]

Using our favorite trigonometric identity again
\[ \sin^2 s = 1 - \cos^2 s \]
\[ = 1 - \frac{1 + \cos 2s}{2} =  \frac{1 - \cos 2s}{2} \]

The final formulas are obtained by taking square roots.  

Leaving this as the square we have that
\[ \sin^2 \frac{\pi}{8} = \frac{1 - \cos \frac{\pi}{4}}{2} \]
\[ = \frac{1}{2} (1 - \frac{1}{\sqrt{2}}) \]

Somehow, it must be true that
\[ \frac{1}{4 + 2 \sqrt{2}} = \frac{1}{2} (1 - \frac{1}{\sqrt{2}}) \]

Factor out $1/2$
\[ \frac{1}{2 + \sqrt{2}} = 1 - \frac{1}{\sqrt{2}} \]

Multiply both sides by $\sqrt{2}$
\[ \frac{1}{\sqrt{2} + 1} = \sqrt{2} - 1 \]

Aha!  A difference of squares:
\[ (\sqrt{2} + 1)(\sqrt{2} - 1) = 2 - 1 = 1  \]

$\square$

\subsection*{geometric derivation of double angle}

Here is a simple geometric derivation of the double angle formula for sine.  We start with an inspired diagram.

\begin{center} \includegraphics [scale=0.4] {double_angle.png} \end{center}

From our work with arcs, we know that angle $\phi$ on the left is one-half the central angle $\theta$, and from Thales' theorem, that the angle on the circle at the top-right (formed by the dotted lines) is a right angle.  

So the small right triangle with hypotenuse $h$ also has angle $\phi$, as labeled, since they have the same complementary angle.

It helps to know where we're going, as well.  From above:
\[ \sin \theta = 2 \sin \phi \cos \phi \]

Just reading off the small triangle we have that
\[ \sin \phi \cos \phi = \frac{x}{h} \cdot \frac{y}{h} \]
\[ = \frac{xy}{h^2} \]

and Pythagoras says that $h^2 = x^2 + y^2$ so:
\[ = \frac{xy}{x^2 + y^2} \]

We're looking to involve $\theta$.  From the right triangle containing that angle:
\[ (1 - x)^2 + y^2 = 1 \]
\[ -2x + x^2 + y^2 = 0 \]
\[ 2x = x^2 + y^2 \]

So, substituting into what we had above:
\[ \sin \phi \cos \phi = \frac{xy}{2x} \]
\[ 2 \sin \phi \cos \phi = y = \sin \theta \]

For the other one, again we remind ourselves of the target:
\[ \cos \theta = \cos^2 \phi - \sin^2 \phi \]

\begin{center} \includegraphics [scale=0.4] {double_angle.png} \end{center}

Substituting, using the diagram:

\[ \cos \theta = \frac{y^2}{h^2} - \frac{x^2}{h^2} \]
\[ = \frac{y^2 - x^2}{h^2} \]
\[ = \frac{y^2 - x^2}{x^2 + y^2} \]

But $2x = x^2 + y^2$:

\[ = \frac{2x -x^2 - x^2}{2x} \]
\[ = 1 - x \]
\[ = \cos \theta \]

$\square$

Similar calculations can be done for a diagram that is slightly relabeled, to provide a simple geometric proof of the Pythagorean theorem.

\begin{center} \includegraphics [scale=0.4] {double_angle_2.png} \end{center}

Thales theorem allows us to deduce that the triangle with two dotted sides is a right triangle, hence the two angles labeled $\phi$ are equal by complementarity with the same angle.  By similar triangles, we have then that
\[ \tan \phi = \frac{y}{r + x} = \frac{r - x}{y} \]
\[ y^2 = r^2 - x^2 \]
\[ x^2 + y^2 = r^2 \]

For the double-angle calculations, there are two triangles with base angle $\phi$ and the hypotenuse a dotted line.  The squared lengths of the two hypotenuses are:
\[ (r + x)^2 + y^2 = r^2 + 2rx + x^2 + y^2 \]
\[ = 2r^2 + 2rx = 2r(r + x) \]
and 
\[ (r - x)^2 + y^2 = r^2 - 2rx + x^2 + y^2 \]
\[ = 2r^2 - 2rx = 2r(r - x) \]

So then the products of $\sin$ and $\cos$ work out to be simple ratios, which in each case will have further cancelations.

As one example, for the large triangle:
\[ \cos^2 \phi - \sin^2 \phi = \frac{(x+r)^2}{2r(r + x)} - \frac{y^2}{2r(r + x)} \]

Let us just work with the numerator for a minute:
\[ (x+r)^2 - y^2 = x^2 + 2xr + r^2 - (r^2 - x^2) \]
\[ 2x^2 + 2xr = 2x(x + r) \]
So we see that $2(x+r)$ cancels and the ratio is just
\[ \frac{x}{r} \]
which is $\cos \theta$.

\subsection*{another calculation}
We found previously that 
\[ \sin \frac{\pi}{4} = \cos \frac{\pi}{4} = \frac{1}{\sqrt{2}} \]
\[ \sin \frac{\pi}{6} = \cos \frac{\pi}{3} = \frac{1}{2} \]
\[ \sin \frac{\pi}{3} = \cos \frac{\pi}{6} = \frac{\sqrt{3}}{2} \]

These angles correspond to 30, 45 and 60 degrees.  It might be nice to have sine and cosine of 15 and 75 degrees as well.  That would make even divisions of the first 90 degrees.  We can get them as the sum and difference of $\pi/4$ and $\pi/6$.

Let $s = \pi/4$ and $t = \pi/6$.  Then

\[ \sin \frac{\pi}{12} = \sin s - t = \sin s \cos t - \sin t \cos s \]
\[ = \frac{1}{\sqrt{2}} \cdot \frac{\sqrt{3}}{2} - \frac{1}{2} \cdot \frac{1}{\sqrt{2}} = \frac{\sqrt{3} - 1}{2 \sqrt{2}} \]
\[ \cos \frac{\pi}{12} = \cos s - t = \cos s \cos t + \sin s \sin t \]
\[ = \frac{\sqrt{3}}{2} \cdot \frac{1}{\sqrt{2}} - \frac{1}{2} \cdot \frac{1}{\sqrt{2}} = \frac{\sqrt{3} + 1}{2 \sqrt{2}} \]

We just check that $\sin^2 \theta + \cos^2 \theta = 1$:
\[ \frac{(\sqrt{3} - 1)^2 + (\sqrt{3} + 1)^2}{(2 \sqrt{2})^2} \]
\[ = \frac{3 - 2 \sqrt{3} + 1 + 3 + 2 \sqrt{3} + 1}{8} = 1 \]

We can calculate similarly for $s + t = 5 \pi/12$ or just switch sine and cosine from $\pi/12$.

\end{document}