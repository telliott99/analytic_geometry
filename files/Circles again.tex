\documentclass[11pt, oneside]{article} 
\usepackage{geometry}
\geometry{letterpaper} 
\usepackage{graphicx}
	
\usepackage{amssymb}
\usepackage{amsmath}
\usepackage{parskip}
\usepackage{color}
\usepackage{hyperref}

\graphicspath{{/Users/telliott/Dropbox/Github-math/figures/}}
% \begin{center} \includegraphics [scale=0.4] {gauss3.png} \end{center}

\title{Circle analytically}
\date{}

\begin{document}
\maketitle
\Large

%[my-super-duper-separator]

We now consider what are called quadratic forms, as distinguished from linear equations (i.e., for lines).  Quadratics contain one or two squared terms (or a term that mixes $x$ and $y$).  
\begin{center} \includegraphics [scale=0.5] {conic_sections.png} \end{center}

One of the simplest examples is the equation for a unit circle centered at the origin:
\[ x^2 + y^2 = 1 \]

Pythagoras tells us that for a point $(x,y)$, the square of the distance from the origin is $x^2 + y^2$.  This equation describes all the points whose distance from the origin is equal to $\sqrt{1} = 1$.  But all  the points equi-distant to a point form a circle.  We generalize
\[ x^2 + y^2 = r^2 \]

It is clear that when $y = 0$, $x = \pm \ r$, and when $x = 0$, $y = \pm \ r$.  $r$ is the radius of the circle.

\subsection*{shifted circle}

If the center is at $(1,0)$, what this amounts to is adding $1$ to the $x$ value of every point.  If we solve for $x$
\[ x = \sqrt{1 - y^2} \]
and then add $1$
\[ x = \sqrt{1 - y^2} + 1 \]
\[ (x - 1)^2 = 1 - y^2 \]
\[ (x - 1)^2 + y^2 = 1 \]

So even though the displacement is positive, say a value $h$, the $x$ term becomes $(x - h)$, which is counter-intuitive.

\begin{center} \includegraphics [scale=0.4] {circles_5.png} \end{center}

Here are two circles, one at the origin with radius $a$, and one at $(h,0)$ with radius equal to $b$.  The equation of the displaced one is:
\[ (x' - 1)^2 + y'^2 = b^2 \]
\[ x'^2 - 2x' + 1 + y'^2 = b^2 \]
and the other is of course:
\[ x^2 + y^2 = a^2 \]

To find the points where the two circles intersect, we must have $x = x'$ and $y = y'$:
\[ x^2 - 2x + 1 + y^2 = b^2 \]
\[ x^2 + y^2 = a^2 \]

Subtract to obtain:
\[ -2x + 1 = b^2 - a^2 \]
\[ 2x - 1 = a^2 - b^2 \]
If $a = b = 1$ then
\[ x = \frac{1}{2} \]
and 
\[ y = \pm \sqrt{1 - x^2} = \pm \frac{\sqrt{3}}{2} \]

\begin{center} \includegraphics [scale=0.4] {circles_5.png} \end{center}

Notice that the distance from the origin to the point of intersection is 
\[ \sqrt{(\frac{1}{2})^2 + (\frac{\sqrt{3}}{2})^2} \]
\[ = 1 \]
But that's just equal to $h$.

We have found the vertex of an equilateral triangle.  Since you remember \hyperref[sec:Euclid1]{\textbf{this}}, so that's no surprise.

For another example, let $a = 2$, and everything else stay the same.  Then 
\[ 2x - 1 = 2^2 - 1^2 \]
\[ 2x = 4 \]
\[ x = 2 \]

So 
\[ y =  \pm \sqrt{r^2 - x^2} \]
\[ = \pm \sqrt{2^2 - 2^2} = 0 \]

And if $a$ were larger than twice $h$, we get to the last step and find the square root of a negative number, because the circles don't intersect.

\begin{center} \includegraphics [scale=0.4] {circles_6.png} \end{center}

\subsection*{tangent to the circle}

\begin{center} \includegraphics [scale=0.4] {tangent6.png} \end{center}

Let's consider an arbitrary point on a circle $(x,y)$.  The radius squared is equal to $x^2 + y^2$.

We draw the radius to the point $(x,y)$ and also the tangent at the same point, and recall an important fact from plane geometry:  the tangent and the radius meet at a right angle.

\emph{Proof}.

Of course, we could just appeal to symmetry.

The tangent is defined as the line that touches only a single point on the circle.  That point is closer to the origin of the circle than any of the other points on the tangent line, since none of them touch the circle and so their distance to the origin is greater than the distance to the tangent point.

We proved in the chapter on right triangles that the distance from a point to a line is least at the point where the new segment makes a right angle with the line.

$\square$

The slope of the radius to that point is just $y/x$.  From our work with lines and slopes we know that perpendicular (orthogonal) lines have slopes whose product is $-1$.  So the slope of the tangent line is $-x/y$.

We use the the point-slope equation for the two points $(x,y)$, and $(x_0,0)$ to write:
\[ -\frac{x}{y} = \frac{0 - y}{x_0 - x} \]
\[ x(x_0 - x) = y^2 \]
\[ xx_0 = r^2 \]
For a unit circle, $x_0 = 1/x$

This makes sense, when $x = r$, then $x_0 = 1/r$ and as $x \rightarrow 0$, $x_0 \rightarrow \infty$.

\begin{center} \includegraphics [scale=0.4] {tangent7.png} \end{center}

Similarly, the line between $(0,y_0)$ and $(x,y)$ gives:
\[ -\frac{x}{y} = \frac{y - y_0}{x - 0} \]
\[ y(y-y_0) = -x^2 \]
\[ r^2 = yy_0 \]
For a unit circle, $y_0 = 1/y$.

You can also do this using the Pythagorean theorem.

\begin{center} \includegraphics [scale=0.4] {tangent8.png} \end{center}

The radius squared is $x^2 + y^2$.  The length of the segment from $(x,y)$ to $(x',0)$ is
\[ (x' - x)^2 + y^2 \]
And the hypotenuse on the bottom is $x'^2$.  So
\[ x'^2 = (x' - x)^2 + y^2 + x^2 + y^2 \]
\[ = x'^2 - 2x'x + x^2 + y^2 + x^2 + y^2 \]
\[ = x'^2 - 2x'x + 2r^2 \]

Thus
\[ 2r^2 = 2x'x \]
\[ rr = x'x \]

\subsection*{displaced circle}

Let's go back to the displaced circle (not centered at the origin.  Generally 
\[ (x - h)^2 + (y - k)^2 = r^2 \]
where the origin of the circle is at $(h,k)$.  

Multiplying out:
\[ x^2 - 2hx + h^2 + y^2 - 2ky + k^2 = r^2 \]
\[ x^2  + y^2 - 2hx - 2ky + (h^2 + k^2 - r^2) \]

Comparing to the most general form for a quadratic
\[ Ax^2 + By^2 + Cxy + Dx + Ey + F = 0 \]

We see that
\[ A = 1, \ \ \ \ B=1, \ \ \ \ C = 0 \]
and in fact, this is true for all circles.  (If $A = B \ne 1$, just divide all the terms by $A$).

Moreover
\[ D = - 2h, \ \ \ \ E = - 2k, \ \ \ \ F = h^2 + k^2 - r^2 \]

This equation can help us solve the following problem from Hamming:  find the equation of the circle that passes through the following three points:
\[ (-1,1), (1,1), (2,3) \]

\begin{center} \includegraphics [scale=0.9] {Hamming_6_2_2.png} \end{center}
We write
\[ x^2 + y^2 + Dx + Ey + F = 0 \]
From the values of $x$ and $y$ at each of the three points we get
\[ 1 + 1 - D + E + F = 0 \]
\[ 1 + 1 + D + E + F = 0 \]
\[ 4 + 9 + 2D + 3E + F = 0 \]
Three equations in three unknowns.  We can do that.

Adding the first two equations together:
\[ 4 + 2(E + F) = 0 \]
so $E + F = -2$.

Subtracting the first two equations (or substituting the result for $E + F$) tells us that $D = 0$.

Adding $(-3)$ times the second equation to the third gives:
\[ 1 + 6 - D - 2F = 0 \]
\[ 7 - 2F = 0 \]
$F = 7/2$, and since $E + F = -2$, $E = -11/2$.

So the solution is
\[ x^2 + y^2 - \frac{11}{2} y + \frac{7}{2} = 0 \]

You can check that it works for all three points:
\[ (-1,1), (1,1), (2,3) \]
The first two are easy, while the third gives
\[ 4 + 9 - \frac{11}{2} 3 + \frac{7}{2} = 0 \]
\[ 8 + 18 - 33 + 7 = 0 \]
which looks correct.

\section*{completing the square}

We can improve this by completing the square.  We see that
\[ y^2 - \frac{11}{2} y + ( \frac{11}{4})^2 = (y - \frac{11}{4})^2 \]

We must add that back to the right-hand side of the original to obtain:
\[ x^2 +  (y - \frac{11}{4})^2 =  ( \frac{11}{4})^2 - \frac{7}{2} \]

The center is at $(0,11/4)$.  The radius doesn't come out cleanly but $r^2$ is
\[ \frac{121}{16} - \frac{56}{16} = \frac{65}{16} \]
so $r$ is slightly more than $2$.

Or recall that we had:
\[ D = - 2h, \ \ \ \ E = - 2k, \ \ \ \ F = h^2 + k^2 - r^2 \]

From this, we have that $h = 0$ and $k = -E/2 = 11/4$, and the radius is more complicated, as we said.

\subsection*{plane geometry}

We can check our work by solving the problem using a technique from plane geometry.  Again, we want the circle passing through three points:
\[ (-1,1), (1,1), (2,3) \]

Take two of the points to be placed on a circle and construct the line segment joining them (a chord of the circle).  Find the midpoint of the chord and erect a perpendicular bisector through the midpoint.  Now, every point lying on the bisector is equidistant from the two starting points.  Proof:  draw the two triangles including that point, the two starting points and the midpoint of the bisector.  The two triangles are congruent.  Here is the general picture.

\begin{center} \includegraphics [scale=0.6] {three_point_circle2.png} \end{center}
It's a bit trickier to prove that \emph{every} point that is equidistant from the two points lies on the bisector.  We assume that.  

Since every point that is equidistant from the two points lies on the bisector, the radius of the circle lies on the bisector.  

Then, erect a perpendicular bisector of a chord joining another pair chosen from the three points.  This new bisector and the first one meet at the center of the circle.

In our case two points $(-1,1), (1,1)$ are symmetric about the $y$-axis.  Therefore it is clear that the perpendicular bisector for these two points is the $y$-axis.
\begin{center} \includegraphics [scale=0.9] {Hamming_6_2_2.png} \end{center}
For the second bisector, form the vector between $(1,1)$ and $(2,3)$ as $\mathbf{v} = \langle 1,2 \rangle$.  The midpoint is at $(1,1) + \mathbf{v}/2 = (3/2, 2)$.

The slope of the bisector is the negative inverse of the slope for the chord which is $- 1/2$ so the equation of the bisector is
\[ y - y_0 = -\frac{1}{2} (x - x_0) \]
Plugging in the point that we know, we obtain
\[ y - 2 = -\frac{1}{2} (x - 3/2) \]
We want to solve for $y$ when $x = 0$, crossing the first bisector, the $y$-axis
\[ y - 2 = -\frac{1}{2} (- 3/2) \]
\[ y = \frac{11}{4} \]
So the center is at $(0,11/4)$, which matches what we had before.  We compute the distance to one of the points $(1,1)$ as
\[ d = \sqrt{1^2 + (11/4 - 1)^2} = \sqrt{1 + 49/16} \]
which also matches our previous result.

\begin{center} \includegraphics [scale=0.8] {Hamming_6_2_3.png} \end{center}

Here is another problem from Hamming.  We need to prove that the angle above is a right angle. We know this is true from geometry.  But now we wish to practice our analytic geometry.

Suppose the equation of the circle is 
\[ x^2 + y^2 = a^2 \]
The point on the circle is $(x_0,y_0)$.

Our first solution uses slopes and points.  The line from $(-a,0)$ to $(x_0,y_0)$ has slope
\[ m_1 = \frac{y_0}{x_0 + a} \]
The line from $(a,0)$ to $(x_0,y_0)$ has slope
\[  m_2 = \frac{y_0}{a - x_0} \]
Two lines meet at a right angle if the product of their slopes is equal to $-1$.
\[ m_1 m_2 = \frac{y_0}{x_0 + a} \ \frac{y_0}{a - x_0} \]
\[ = \frac{y_0^2}{a^2 - x_0^2} = \frac{y_0^2}{x_0^2 + y_0^2 - x_0^2} = - 1 \]

This was not pretty, it's just good exercise.  

And here is a proof using vectors and the dot product.  Consider the semicircle centered on the origin with radius $a$, so the ends of the diameter are at $(x = \pm \ a, 0)$.  

Form the vectors from those ends to an arbitrary point $(x,y)$ on the perimeter:
\[ \mathbf{u} = \ \langle x + a, y \rangle \]
\[ \mathbf{v} = \ \langle x - a, y \rangle \]
Notice that
\[  \mathbf{u} \cdot  \mathbf{v} = (x + a)(x - a) + y^2 \]
\[ = x^2 -a^2 + y^2 = 0 \]
because $x^2 + y^2 = a^2$ for any point on the circle.

As our last example, consider the problem of finding the equation of a line tangent to a circle that goes through some arbitrary point $b$.
\begin{center} \includegraphics [scale=0.4] {Hamming_6_3_1_rev.png} \end{center}

We take the circle to have radius $a$ and be centered at the origin.  We take the point $b$ to be on the $y$-axis.  The equation of the line on the right side is
\[ \frac{y - y_0}{x - x_0} = m = \frac{y - b}{x} \]
\[ y = mx + b \]
(well, of course).

For the point or points where the line intersects the circle we also have
\[ y = \sqrt{a^2 - x^2} \]
\[ \sqrt{a^2 - x^2} =  mx + b \]
\[ a^2 - x^2 = m^2x^2 + 2bmx + b^2 \]
\[ (m^2 + 1)x^2 + 2bmx + b^2 - a^2 = 0 \]
From the quadratic equation:
\[ x = \frac{-2bm \pm \sqrt{4b^2m^2 - 4(m^2 + 1)(b^2 - a^2)}}{2(m^2 + 1)} \]
We are looking for the case where there is a single solution so the discriminant under the square root must be equal to zero:
\[ 4b^2m^2 = 4(m^2 + 1)(b^2 - a^2) \]
\[ m^2b^2 = m^2b^2 - m^2a^2 + b^2 - a^2 \]
\[ 0 = -m^2a^2 + b^2 - a^2 \]
\[ m = \pm  \frac{\sqrt{b^2 - a^2}}{a}  \]

This makes sense since if $a=b$ the single tangent should be horizontal with zero slope.  Notice that if $a^2 > b^2$ there is no real solution.  This corresponds to having $b$ inside the circle.

\subsection*{one more}

And then lastly, I found this problem on the web:

\begin{center} \includegraphics [scale=0.3] {circle_prob.png} \end{center}

The challenge was, can you immediately write the equation of this circle?  I said sure
\[ (x - h)^2 + (y - k)^2 = r^2 \]
\[ (x - 2.5)^2 + (y - 4)^2 = 1.8^2 \]

But notice we are given $A = (1,3)$ and $B = 4,5)$. 

There is a reason!  According Thales' theorem, every point $(x,y)$ on the circle should have a vector to $A$ and a vector to $B$, that when multiplied to give the dot product, you get zero.

Write:
\[ \langle (x - 1), (y - 3) \rangle \cdot \langle (x - 4),(y-5)\rangle = 0 \]
\[ x^2 - 5x + 4 + y^2 - 8y + 15 = 0 \]

We need to complete \emph{two} squares:
\[ x^2 - 5x + 2.5^2 + y^2 - 8y + 4^2 + ... \]

$2.5^2 = 6.25$ and $4^2 = 16$ and the sum is $22.25$.  Since we originally had $19$ we now have $-3.25$ on the left and write
\[ (x - 2.5)^2 + (y - 4)^2 = 3.25 \]

$\sqrt{3.25} \ne 1.8$.  What went wrong?

Calculate the distance between $A$ and $B$ as $\sqrt{3^2 + 2^2} = \sqrt{13} = 3.60555 ...$.  Half of that is the radius, which is $1.802775...$.

And that matches $\sqrt{3.25} = 1.802775...$.

The problem was that although the upper edge of the circle looked to be at $4 + 1.8 = 5.8$, that is not exactly correct.  The statement was that $A$ and $B$ have integer values for $(x,y)$.

\end{document}