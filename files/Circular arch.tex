\documentclass[11pt, oneside]{article} 
\usepackage{geometry}
\geometry{letterpaper} 
\usepackage{graphicx}
	
\usepackage{amssymb}
\usepackage{amsmath}
\usepackage{parskip}
\usepackage{color}
\usepackage{hyperref}

\graphicspath{{/Users/telliott/Github/figures/}}
% \begin{center} \includegraphics [scale=0.4] {gauss3.png} \end{center}

\title{Circular segment}
\date{}

\begin{document}
\maketitle
\Large

%[my-super-duper-separator]

I found a hard geometry problem on the web:
\begin{center} \includegraphics [scale=0.5] {circ_seg_prob.png} \end{center}
We know it's hard because it says so!

I liked it particularly because there are several different ways to calculate the answer using basic geometry and trigonometry, plus standard integration as well as polar integration.  I got the same answer each time, fortunately.

In this book we just look at the geometry.

As a first step, observe that the problem has been made artificially complicated by using these particular values for the side lengths.  If both dimensions are scaled down by a factor of 5, then we obtain a rectangle with side lengths $4$ and $2$, and the two circles become unit circles.  

We must just remember to re-scale to obtain the final answer, multiplying the final area by $25$.

The area of any arched corner segment is pretty easy, since $4$ of them put together are equal to the difference between the area of a square with side length $2$ and a unit circle:  $4 - \pi$, so each one is $1 - \pi/4$.

The real difficulty is the upper right-hand corner.  We ignore the above and just concentrate on this part of the problem.

\begin{center} \includegraphics [scale=0.25] {circ_seg_prob2.png} \end{center}

One of the arches is divided into two pieces, and we are supposed to only count the red part of the arch.

The basic right triangle that we see repeated in these images has side lengths in the ratio $1:2$.  Its area is just $1$, and the smaller angle is 
\[ \phi = \tan^{-1} 0.5 \approx 0.4636 \text{ rad } \approx  26.565 \text{ deg } \]

\begin{center} \includegraphics [scale=0.4] {circ_seg3.png} \end{center}
That's not a nice round number, but OK.
 
My first thought was to calculate the area cut off by the chord of a circle, called a "circular segment".  Then we could calculate the white part of the divided arch:
\[ \text{ triangle } - \text{ segment }  - \text{ arch } \]
\[ 1 - \text{ segment }  - (1 - \pi/4) \]

and subtract that from one whole arch to get the red part.
\[ (1 - \pi/4) - \ [ \ 1 - \text{ segment }  - (1 - \pi/4) \ ] \]
\[ 1 -  \frac{\pi}{2}  + \text{ segment } \]

We will use this result at the end for our final answer.  

There is an easier way, which we find by exploring this direction just a little further.

A circular segment is like a polar cap, but in two dimensions.
\begin{center} \includegraphics [scale=0.6] {circ_seg.png} \end{center}

\url{http://mathworld.wolfram.com/CircularSegment.html}

We carefully distinguish between the circular segment, in yellow, and the circular sector, which is the area of that slice of the circular pie swept out by the angle $\theta$.  

The area of the circular sector with central angle $\theta$ is the fraction of the total circular angle, times the area of a unit circle.  The result is just half the angle.
\[ \frac{\theta}{2 \pi} \cdot \pi = \frac{\theta}{2} \]

For the actual calculation of a circular segment, we would need not only the angle $\theta$, but also $r$ and $a$, which we would need to derive from $\theta$ by applying the Pythagorean theorem and/or trigonometry.  It can be done!  However, we see a better way.  

\subsection*{isosceles triangles}

The key idea is to draw another radius on our diagram, and realize that $\theta$ is the apex angle of an isosceles triangle.  The two sides are both radii of our circle, and so are equal to each other!  Therefore $\theta = \pi - 2 \phi$.

\begin{center} \includegraphics [scale=0.4] {circ_seg5.png} \end{center}

We recall that this is a basic theorem from the geometry of circles:  the central angle subtending a given arc is twice the measure of the angle on the periphery.

Now we're ready.

The area of the circular segment with a central angle $\theta$ is
\[ A = \frac{\theta}{2 \pi} \cdot \pi R^2 \]

This is a unit circle so $R = 1$ and then
\[ A = \frac{\theta}{2} \]
Since the angle we're talking about is actually $2 \phi$ we obtain finally
\[ A = \phi \]
That's for the area of the circular segment.

For the triangles, we see that $r$ is the bisector of an isosceles triangle.  Therefore
\[ \frac{r}{R} = r = \sin \phi \]
and 
\[ \frac{a}{R} = a = \cos \phi \]
So the area of the two triangles is
\[ \sin \phi \cos \phi \]

and the area of the red part of the arch is then the area of the large right triangle ($1/2 \cdot 1 \cdot 2 = 1$) minus the two areas we just calculated:
\[ 1 - \phi - \sin \phi \cos \phi \]

One could change the angle $\phi$ and this result would still be valid, except that the area of the large triangle would change.

The base would still be $2$, and the height would be $2 \tan \phi$, so the area would be $2 \tan \phi$ rather than $1$.

\subsection*{using the tangent}

This particular angle has tangent $1/2$, which gives an additional simplification:
\[ \frac{1}{2} = \frac{\sin \phi}{\cos \phi} \]
\[ 2 \sin \phi = \cos \phi \]
\[ 2 \sin^2 \phi = \sin \phi \cos \phi \]

We can do even better than that.
\[ 2 \sin \phi = \cos \phi \]
\[ 4 \sin^2 \phi = \cos^2 \phi \]
so
\[ \sin^2 \phi + \cos^2 \phi = 1 \]
\[ 5 \sin^2 \phi = 1 \]
\[ 2 \sin^2 \phi = \frac{2}{5} \]

Our final answer for this part is
\[ 1 - \phi - \frac{2}{5} \]

To get the answer required by the original problem statement, we must add the area of three arches, $3 \cdot (1 - \pi/4)$, and remember to scale the answer back up by a factor of $25$.

\end{document}