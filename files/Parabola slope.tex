\documentclass[11pt, oneside]{article} 
\usepackage{geometry}
\geometry{letterpaper} 
\usepackage{graphicx}
	
\usepackage{amssymb}
\usepackage{amsmath}
\usepackage{parskip}
\usepackage{color}
\usepackage{hyperref}

\graphicspath{{/Users/telliott/Dropbox/Github-Math/figures/}}
% \begin{center} \includegraphics [scale=0.4] {gauss3.png} \end{center}

\title{Parabola}
\date{}

\begin{document}
\maketitle
\Large

%[my-super-duper-separator]

\subsection*{direction of opening}

We continue looking at parabolas, the graphs of equations like $y = ax^2$, this time emphasizing the point of view of analytical geometry.

Any simple parabola (with no terms that mix $x$ and $y$), has its axis of symmetry parallel to either the $x$- or $y$-axis.

Below we have $y = ax^2$ with either $a > 0$ (left panel) or $a < 0$ (right panel)
\begin{center} \includegraphics [scale=0.30] {para9.png} \end{center}

And then we have $x = ay^2$ with either $a > 0$ (left panel) or $a < 0$ (right panel)
\begin{center} \includegraphics [scale=0.30] {para10.png} \end{center}

In any case, we can just switch$-a$ and $a$, or $x$ and $y$ as needed, to obtain a figure like the first one, left panel.  We focus on  parabolas pointing up, with no loss of generality.  

Those containing terms like $Bxy$, we leave aside as a complication to be returned to later.

\subsection*{vertex at different points}

Start with a parabola having its vertex at the origin.  The equation will be simply $y = ax^2$.  

$a$ is called the \emph{shape factor}.  It governs how steeply the curve rises (and as we said, by its sign it determines in which direction it opens).

We can see that the vertex is at $(0,0)$, because (i) $(0,0)$ is on the curve and (ii) $y \ge 0$, so $0$ is the smallest $y$.

It turns out that any parabola with its vertex at a point other than the origin, can be described by the formula
\[ y - k = a(x - h)^2 \]

Moving the graph amounts to changing the values of $x$ and $y$ for every point on the curve.

For example, to move the vertex up by two units to $(0,2)$, add $2$ to the value of $ax^2$ for every $x$.  The result is
\[ y = ax^2 + k \]

where $k$ is the amount of vertical shift.  This can be rearranged to 
\[ (y - k) = ax^2 \]

The vertex is at $y = 2$ (positive), but the formula says to subtract $k$ from $y$, which is a bit counter-intuitive.

Changes in $x$ are taken into account \emph{before} squaring.  For a parabola whose vertex is on the $x$-axis, the formula becomes
\[ y = a(x-h)^2 \]

If you try this for a vertex at $(1,0)$, and plot the values, you will see that this is correct.  For example, $x$ values symmetric on each side of the vertex yield the same $y$, in the proportion $a(\Delta x)^2$, where $\Delta x = x - h$.

So the general formula for a parabola with its vertex at the point $(h,k)$ is
\[ y - k = a(x - h)^2 \]

If we work with this a bit, multiplying out:
\[ y - k = a(x^2 - 2xh + h^2) \]
\[ y = ax^2 - 2ah x + ah^2 + k \]

In this form the cofactors are usually simplified as
\[ y = ax^2 + bx + c \]
which we are used to seeing from algebra.

Comparing the two, we see that the cofactors of the $x$ term must be equal:
\[ -2ahx = bx \]
\[ h = -\frac{b}{2a} \]

and the constant terms must be equal as well
\[ c = ah^2 + k \] 
\[ k = c - ah^2 \]
\[ = c - \frac{b^2}{4a} \]

We can check this as follows.  The first equation is commonly used to find the vertex for a given parabola.  The $x$-value of the vertex is $h = -b/2a$.

Then the $y$-coordinate ($k$) can obtained by plugging into the given equation:
\[ k = a(-\frac{b}{2a})^2 + b(-\frac{b}{2a}) + c \]
\[ k - c = \frac{b^2}{4a} - \frac{b^2}{2a} \]
\[ = - \frac{b^2}{4a}  \]
which matches what we had above.

\subsection*{roots}

Probably the most common thing we're asked to do with a quadratic equation like this is to find the roots.  These are the values of $x$ for which $y=0$ is a solution.  They are the points where the graph of the curve crosses the $x$-axis.

By trying different possibilities it becomes clear that it is possible to have 0, 1 or 2 roots.  

In the figure below, the black curve has two roots, the red curve has one.  The latter's equation is $y = x^2$, the former is $y = x^2 - 1$ and then we can see that $x^2 = 1$ has two real solutions $x = \pm 1$.

\begin{center} \includegraphics [scale=0.50] {para8.png} \end{center}

On the left, the magenta curve does not cross the $y$-axis.  Its equation is $y = x^2 + 1$, and there are no (real) solutions, no values of $x$ that solve the equation when $y = 0$.
\[ 0 = x^2 + 1 \]
\[ x^2 = - 1 \]

To find the roots of
\[ ax^2 + bx + c = 0 \]
We can guess solutions by trying to factor into a form like:
\[ (x - s)(x - t) = 0 \]

The case of a single root occurs when $s = t$ so we have $a(x - s)^2 = 0$.  A common example of that is a parabola with its vertex at the origin, so $s = 0$ and $y = ax^2$ (right panel, above).

Roots do not have to be integers (or even rational).  An arguably more productive and certainly more general approach to finding them is the process of \emph{completing the square}.  

\subsection*{completing the square}
Suppose we have
\[ y = ax^2 + bx + c \]
When $y = 0$:
\[ ax^2 + bx + c = 0 \]

First, multiply through by $1/a$ and place the constant term on the right-hand side:
\[ x^2 + \frac{b}{a} x = - \frac{c}{a} \]

We want to write the left-hand side
\[ x^2 + \frac{b}{a} x \]

as a perfect square.  Something like
\[ (x + p)^2 = x^2 + 2xp + p^2 \]
What should $p$ be?

If we compare the cofactor of the $x$ term, in our problem we have $b/a$ and in the example we have $2p$.  So $2p = b/a,  \ p = b/2a$ and then $(x + p)$ is like $(x + b/2a)$ so finally $(x + p)^2$ is like
\[ (x + \frac{b}{2a})^2 = x^2 + \frac{b}{a}x + (\frac{b}{2a})^2 \]

The key insight is that on the original left-hand side (three equations back)
\[ x^2 + \frac{b}{a} x \]

the third term is missing, but \emph{we can fix it}.  To maintain equality, simply add the same thing on both sides:
\[ x^2 + \frac{b}{a} x + (\frac{b}{2a})^2  = -\frac{c}{a} + (\frac{b}{2a})^2 \]

So now the left-hand side is a perfect square:
\[ (x + \frac{b}{2a})^2 = -\frac{c}{a} + (\frac{b}{2a})^2 \]
\[ x + \frac{b}{2a} = \pm \sqrt{-\frac{c}{a} + (\frac{b}{2a})^2} \]

Now, we just do a bit of rearranging.

Multiplying top and bottom of the first term under the square root gives a common factor of $4a^2$:
\[ x + \frac{b}{2a} = \pm \sqrt{-\frac{4ac}{4a^2} + (\frac{b}{2a})^2} \]

which can come out of the square root and then matches what's in the second term on the left-hand side:
\[ x + \frac{b}{2a} = \pm \frac{\sqrt{-4ac + b^2}}{2a} \]
which we rearrange slightly to give the standard \emph{quadratic formula}:
\[ x = \frac{-b \pm \sqrt{b^2 - 4ac}}{2a} \]

This formula always works to find the roots of an equation, if they exist.  The quantity under the square root is called the discriminant
\[ D = b^2 - 4ac \]
If $D < 0$ then $\sqrt{D}$ does not exist in the real numbers and there is no $x$ such that $y = 0$.  That corresponds to the case where the parabola does not cross the $x$-axis.

If $D = 0$ then there is a single root, and the graph just touches the $x$-axis.

\subsection*{check the answer}

We assert that when $x$ takes on those two values, it is a solution for
\[ ax^2 + bx = -c \]
Take the positive root:
\[ x = \frac{1}{2a} \ [ -b + \sqrt{b^2 - 4ac} \ ] \]
Compute $ax^2$
\[ ax^2 = \frac{1}{4a} \ [ b^2 - 2 b \sqrt{b^2 - 4ac} + b^2 - 4ac \ ] \]
\[ = \frac{1}{4a} \ [ 2b^2 - 2b \sqrt{b^2 - 4ac} - 4ac \ ] \]

and then for $bx$
\[ bx = \frac{1}{2a} \ [ -b^2 + b \sqrt{b^2 - 4ac} \ ] \]
multiply by $2$ on top and bottom on the right-hand side:
\[ bx = \frac{1}{4a} \ [ -2b^2 + 2b \sqrt{b^2 - 4ac} \ ] \]

All the terms of $bx$ cancel terms in $ax^2$.  What's left is $- 4ac$ so then
\[ \frac{1}{4a} \ (- 4ac) = -c \]

If we had started with the negative root, the square roots in both $ax^2$ and $bx$ would each change sign, but they would still be opposite signs and hence, cancel.

Although the form given above is the standard one
\[ ax^2 + bx + c = 0 \]

there is another.  Since we're after roots, divide by $a$ so we can get rid of one cofactor
\[ x^2 + px + q = 0 \]

Now the formula gives
\[ x = - \frac{p}{2} \pm \ \sqrt{\frac{p^2}{4} - q} \]

\emph{Proof}.

\[ (x + \frac{p}{2})^2 = x^2 + pz + \frac{p^2}{4} \]
\[ = \frac{p^2}{4} - q \]

Take the roots and subtract $p/2$.

\subsection*{tangent lines}

Consider a parabola and a line on the same graph.  There are four possibilities for the intersection of the line and the parabola.  First and second:  two points (left panel), and no points (right panel).

\begin{center} \includegraphics [scale=0.40] {para31.png} \end{center}

The other two possibilities both have a single point of intersection.  The tangent line at a point (left panel), and a vertical line (right panel).

\begin{center} \includegraphics [scale=0.40] {para32.png} \end{center}

We are particularly interested to know the equations of tangent lines to the parabola, which includes their slopes.  Consider the simplest example:  $y = x^2$.

The point $(1,1)$ is on the curve, because $(x = 1, y = 1)$ satisfies the equation $y = x^2$.

\begin{center} \includegraphics [scale=0.50] {para11.png} \end{center}

Suppose we know that the slope of the tangent to the curve at the point $(1,1)$ is equal to $2$.

The equation of the tangent line is
\[ y - y' = m(x - x') \]
Plugging in for $(x', y') = (1,1)$:
\[ y - 1 = 2(x - 1) \]
\[ y = 2x - 1 \]

Now suppose that we knew only the parabola and this slope, but we did not know the point where the tangent meets the curve, and so do not know the $y$-intercept.

We have the equation of a line:
\[ y = 2x + y_0 \]

We seek points which are simultaneously on the line and the curve.  They must satisfy both equations.

Since this is a tangent line, we seek the value for which this expression has only a single solution.  The tangent "touches" the curve at a single point.

Substitute for $y$ from the equation for the curve:
\[ x^2 = 2x + y_0 \]
\[ x^2 - 2x - y_0 = 0 \]

Use the quadratic formula to set up an expression for $x$:
\[ x = \frac{-b \pm \sqrt{b^2 - 4ac}}{2a} \]

There is a single solution when the part under the square root (the discriminant) is equal to zero.

\[ b^2 - 4ac = 0 \]
\[ b^2 = 4ac \]
\[ (-2)^2 = 4(-y_0) \]
\[ y_0 = -1 \]
Therefore, the equation of the tangent line is $y = 2x - 1$, which matches what we had before.

$y = 2x + y_0$ is a \emph{family} of lines.  For $y_0 = -1$, there is a single solution for $x$ to be both on the line and the parabola.  For $y_0 < -1$, there are no solutions, while for $y_0 > -1$ there are two solutions, because the line actually traces out a chord or secant of the parabola, passing through the curve at two points.

The general solution is as follows:
\[ y = ax^2 \]
\[ y = mx + y_0 \]
The point(s) of intersection are given by $(x,y)$ satisfying both equations:
\[ ax^2 = y = mx + y_0 \]

Then
\[ ax^2 - mx - y_0 = 0 \]
From the quadratic equation
\[ x = \frac{m \pm \ \sqrt{m^2 + 4ay_0}}{2a} \]

For the case of the tangent line, there is a single solution, which happens when the discriminant is zero and then
\[ x = \frac{m}{2a} \]
\[ m = 2ax \]

As we've been saying.  The slope of the tangent to the parabola at a point $(x, ax^2)$ is equal to $2ax$.

We can find the equation of the line by finding $y_0$, the value of $y$ when $x = 0$.
\[ m = 2ax = \frac{y - y_0}{x - 0} \]
\[ 2ax^2 = ax^2 - y_0 \]
\[ y_0 = -ax^2 \]

Note:  do not make the mistake of writing the equation of the line now as
\[ y = mx + y_0 = 2ax \cdot x - ax^2 \]
This is wrong because $m$ and $y_0$ were determined for a particular point on the parabola, but in the equation of the line $y = mx + y_0$, \emph{that} $x$ is any $x$.  Going back to the prime notation we should write:
\[ y = mx + y_0 = 2ax' \cdot x - ax'^2 \]
where $x$ is a variable and $x'$ is a constant.

\subsection*{Kline 4-23}

The slope of the parabola has some simple interesting properties.

For example, pick two points $(x,y)$ and $(x',y')$ on such that the line joining them goes through the focus.
\begin{center} \includegraphics [scale=1.0] {Kline_4_23.png} \end{center}

The tangents through these two points form a right angle, as shown in the figure.

\emph{Proof}.

Recall that when the vertical $DP$ is extended downward, the angle that it makes with the tangent, $\phi$, is equal to the angle $DP$ itself makes with the tangent.  

Furthermore, $FP$ makes the same angle with the tangent, by the "headlight property".  The angle of incidence is equal to the angle of reflection.

The situation is this:

\begin{center} \includegraphics [scale=0.25] {Kline_4_23_b.png} \end{center}

Since $DP$ is parallel to $D'P$, by alternate interior angles the angle $D'QF$ has measure $2 \phi$.  Using the headlight property again we have that $\theta = \theta$.  

Since $2 \phi + 2 \theta$ is 180 degrees, $\theta$ and $\phi$ are complementary, so the angle where the tangents meet is a right angle.

$\square$

\subsection*{further comment}

Next, pick any two points $(x,y)$ and $(x',y')$ on our standard parabola.

The slope of the line that connects those two points is equal to the slope of the parabola at the point whose $x$-value is halfway in between.  
\begin{center} \includegraphics [scale=0.4] {para19.png} \end{center}

For the first part:
\[ m = \frac{y'-y}{x'-x} \]
\[ = \frac{ax'^2 - ax^2}{x'-x} \]
\[ = a \ [ \ \frac{x'^2 - x^2}{x' - x} \ ] \]
\[ = a(x' + x) \]

For the midpoint
\[ x_m = \frac{1}{2} (x' + x) \]
and the slope is
\[ 2a \cdot \frac{1}{2} (x' + x) \]
\[ = a(x' + x) \]

A similar result is that if we pick any two points $(x,y)$ and $(x',y')$, and draw their slopes, the point where the two slope lines meet has its $x$-value exactly halfway in between $x$ and $x'$.

\end{document}