\documentclass[11pt, oneside]{article} 
\usepackage{geometry}
\geometry{letterpaper} 
\usepackage{graphicx}
	
\usepackage{amssymb}
\usepackage{amsmath}
\usepackage{parskip}
\usepackage{color}
\usepackage{hyperref}

\graphicspath{{/Users/telliott/Github/figures/}}
% \begin{center} \includegraphics [scale=0.4] {gauss3.png} \end{center}

\title{Adventitious angles}
\date{}

\begin{document}
\maketitle
\Large

%[my-super-duper-separator]

Here is a problem that looks like it will be easy, and then turns out, famously, to be hard.  Let's take a look.
\begin{center} \includegraphics [scale=0.4] {Langley_1.png} \end{center}

Using theorems about supplementary and vertical angles, and triangle sums, we can fill in some of the values:
\begin{center} \includegraphics [scale=0.4] {Langley_2.png} \end{center}

However, we're not any closer to $x$.  One idea is to use algebra:
\begin{center} \includegraphics [scale=0.4] {Langley_3.png} \end{center}

So now we can use the triangle sum of angles theorem to write some equations:
\[ a + x = 140 \]
\[ a + b = 160 \]
\[ b + c = 130 \]
\[ c + x = 110 \]

Four equations in four unknowns.  That could work.  However, subtract the first from the last and the second from the third to obtain:
\[ a - c = 30 \]
\[ a - c = 30 \]

In linear algebra, we call that degenerate.  The system is not independent and we cannot solve it.

However, we notice some isosceles triangles.  I found three:

\begin{center} \includegraphics [scale=0.4] {Langley_4.png} \end{center}

This will bring us to the "bright idea."
\begin{center} \includegraphics [scale=0.4] {Langley_5.png} \end{center}

We pick another point on one of the sides and draw the two line segments so that the angle labeled at the top is $40^\circ$. 
\begin{center} \includegraphics [scale=0.4] {Langley_6.png} \end{center}

So then we can do some geometry.  First of all, the total angle at vertex $A$ is 80.  Therefore, the angle $DAE$ is 20.  Since the angle at vertex $D$ is 80, $\triangle ADE$ is isosceles and $AD = AE$.

We already knew that $\triangle ABD$ is isosceles, so $AD = AB$.  Thus, $AE = AB$.  So now $\triangle ABE$ is isosceles, and since angle $BAE$ is 60, the triangle is also equilateral.  So $BE = AE = AB$.

By equal base angles $\triangle ACE$ is isosceles, so $AE = EC$.

By adding up angles, we can find that angle $BEC$ equals 40.  Since $\triangle BEC$ is isosceles, $x + 40 = (1/2) 140 = 70$ and then $x = 30$.

There is a lot more.

\url{https://en.wikipedia.org/wiki/Langley’s_Adventitious_Angles}











\end{document}