\documentclass[11pt, oneside]{article} 
\usepackage{geometry}
\geometry{letterpaper} 
\usepackage{graphicx}
	
\usepackage{amssymb}
\usepackage{amsmath}
\usepackage{parskip}
\usepackage{color}
\usepackage{hyperref}

\graphicspath{{/Users/telliott/Dropbox/Github-Math/figures/}}
% \begin{center} \includegraphics [scale=0.4] {gauss3.png} \end{center}

\title{Six functions}
\date{}

\begin{document}
\maketitle
\Large

%[my-super-duper-separator]
\subsection*{basic definitions}
The most elementary trigonometric functions are the sine and cosine.  These are defined in  geometry as ratios of the lengths of the sides of a right triangle.  

Looking at the left panel, we say that the sine of the angle $\alpha$ is the ratio \emph{opposite-over-hypotenuse}, while the cosine of $\alpha$ is the ratio \emph{adjacent-over-hypotenuse}.  Tangent is the ratio \emph{opposite-over-adjacent}. The names are abbreviated to three letters in formulas.  

\includegraphics [scale=0.3] {sine_cosine.png}
\includegraphics [scale=0.5] {sine_cosine_tangent.png}

Using the notation for the sides from the figure:
\[ \sin \alpha = \frac{a}{h}, \ \ \ \ \ \cos \alpha = \frac{b}{h}, \ \ \ \ \ \tan \alpha = \frac{a}{b} = \frac{\sin \alpha}{\cos \alpha} \]

The "unit circle" is a circle of radius $1$ with its center positioned at the origin of coordinates, the place where the $x$ and $y$ axes cross.  From the right panel of the diagram you can see that any point $(x,y)$ on the unit circle can be described in radial coordinates as 
\[ x = \cos \theta \ \ \ \ y = \sin \theta \]
In the diagram, all three right triangles are similar because the red line is an altitude of the largest right triangle.  Thus, by similar triangles, the blue side has this relationship
\[ \frac{\text{blue side}}{1} = \frac{\sin \theta}{\cos \theta} \]
which explains why it is labeled as $\tan$ ($\alpha$).

If the vertex labeled $B$ is denoted angle $\beta$ (the complementary angle of $\alpha$), then the notions of opposite and adjacent switch so that:
\[ \sin \alpha = \cos \beta, \ \ \ \ \ \cos \alpha = \sin \beta \]

If the circle has radius $r$ then
\[ x = r \cos \theta  \ \ \ \  y = r \sin \theta \]

Stewart:

\begin{quote}
The mathematicians of ancient India built on the Greek work to make major advances in trigonometry. They [used] the sine (sin) and cosine (cos) functions, which we still do today. Sines first appeared in the Surya Siddhanta, a series of Hindu astronomy texts from about the year 400, and were developed by Aryabhata in Aryabhatiya around 500. Similar ideas evolved independently in China.
\end{quote}

The other functions are the inverses of sine, cosine and tangent, namely:  cosecant, secant and cotangent.  The secant (inverse cosine) comes up somtimes, but the other two are not especially important in calculus.  

However, there is one context that we will look at, namely, Archimedes determination of the value of $\pi$.  The crucial step in that approach will turn out to be the calculation of the cotangent of the half-angle $\theta/2$ given the values of cotangent and cosecant for angle $\theta$.

The main relationship or identity is derived from the Pythagorean theorem.  We had above that for a unit circle
\[ x = r \cos \theta  \ \ \ \  y = r \sin \theta \]

Since $x$ and $y$ are the sides of a right triangle whose hypotenuse is $r$
\[ x^2 + y^2 = r^2 \]
and for a unit circle
\[ \cos^2 \theta + \sin^2 \theta = 1 \]

which is usually written
\[ \sin^2 \theta + \cos^2 \theta = 1 \]

and transformed (dividing by the cosine squared) to
\[ 1 + \tan^2 \theta = \sec^2 \theta \]

\subsection*{particular values}
We can easily determine the values for these functions for three special cases.  

The first is the angle $45$ degrees or $\pi/4$.  Draw an isosceles right triangle with sides of length $1$ (left panel).

\begin{center} \includegraphics [scale=0.4] {30_45_60.png} \end{center}

Then the hypotenuse has length $\sqrt{2}$ (from Pythagoras) and the values are
\[ \sin \frac{\pi}{4} = \frac{1}{\sqrt{2}} = \cos \frac{\pi}{4} \]
\[ \tan \frac{\pi}{4} = 1 \]

For the other two, bisect an equilateral triangle and erase one half (right panel).  The original vertex angle is $60$ degrees or $\pi/3$, and its complement is the smaller angle, i.e. $30$ degrees or $\pi/6$.

The values are
\[ \sin \frac{\pi}{6} = \frac{1}{2} = \cos \frac{\pi}{3}, \ \ \ \ \ \ \cos \frac{\pi}{6} = \frac{\sqrt{3}}{2} = \sin \frac{\pi}{3} \]
\[ \tan \frac{\pi}{6} = \frac{1}{\sqrt{3}} \]

We can verify that 
\[ ((\frac{1}{\sqrt{2}})^2 + (\frac{1}{\sqrt{2}})^2 = 1, \ \ \ \ \ \  \frac{1}{2})^2 + (\frac{\sqrt{3}}{2})^2 = 1  \]

I find the bisection of an equilateral triangle is an intuitive way to get to a 30-60-90 triangle, and the 45 degree right triangle obvious.  However, you may like this mnemonic device:

\begin{center} \includegraphics [scale=0.2] {sine_table.jpg} \end{center}

with its smooth progression:  $\sqrt{0}$, $\sqrt{1}$, $\sqrt{2}$, $\sqrt{3}$, $\sqrt{4}$.

\subsection*{graph}

\begin{center} \includegraphics [scale=0.4] {sine_cosine_wikipedia.png} \end{center}

Savov:

\begin{quote}The sine function represents a fundamental unit of vibration. The graph of $\sin(x)$ oscillates up and down and crosses the $x$-axis multiple times. The shape of the graph of $\sin(x)$ corresponds to the shape of a vibrating string.\end{quote}

Imagine a circle placed to the left of a graph.  We can think of the sine function as the vertical "shadow" of the $y$-value of the point ($x,y$) as it travels around the circle at the same constant speed as the point on the graph "moves" to the right.  Similarly, the cosine is the shadow of the $x$-value (on the $x$-axis).

\subsection*{visualization of all six functions}

We'll work with a unit circle.  Draw the radius of length $1$ to form the angle $\theta$ and then draw the vertical and horizontal components which we know are $\sin \theta$ and $\cos \theta$. 
\begin{center} \includegraphics [scale=0.4] {six_funcs3.png} \end{center}

By extending the $x$-axis, we can draw another triangle  similar to the first one. If we temporarily say that the line segment colored blue has length $t$, we can see that it is the opposite side in a right triangle with angle $\theta$ where the adjacent side has length $1$.

\[ \frac{t}{1} = \tan \theta \]
Hence $t$ is equal to $\tan \theta$.

We can also modify the figure to get the inverse functions as lengths.

\begin{center} \includegraphics [scale=0.4] {six_funcs4.png} \end{center}

The angle at the top of the figure, labeled in magenta, is also $\theta$.  The reason is that both angles labeled $\theta$ are complementary to the same angle in a right triangle, labeled $\theta'$.

The length in magenta labeled $\cot$ (cotangent) is the adjacent side in a right triangle with opposing side of length $1$ and an angle of $\theta$, hence it is equal to $\cot \theta$.

The length in green is the hypotenuse in this same right triangle.  If we temporarily call this length $c$, we can write that

\[ \frac{\cot \theta}{c} = \frac{\cos \theta}{c \sin \theta} = \cos \theta \]
so that $c$ is equal to $1/\sin \theta$ or the cosecant.

Finally, the secant is the long length along the $x$-axis.  If we temporarily label this length as $s$, then we have that 
\[ \frac{s}{\csc \theta} = \frac{s}{1/\sin \theta} =  \frac{\sin \theta}{\cos \theta} \]
so that $s$ is equal to $1/\cos \theta$ or the secant.

\subsection*{another way}

Here is another version that is similar, from Proofs without words:
\begin{center} \includegraphics [scale=0.3] {six_funcs_6.png} \end{center}

This gives various identities simply, like
\[ 1 + \tan^2 \theta = \sec^2 \theta \]

Our source points out that we can also obtain:
\[ (1 + \tan \theta)^2 + (1 + \cot \theta)^2 = (\sec \theta + \csc \theta)^2 \]

I've certainly never seen that last one, but we can try to prove it algebraically in reverse.
\[ 1 + 2 \tan \theta + \tan^2 \theta + 1 + 2 \cot \theta + \cot^2 \theta = \sec^2 \theta + 2 \sec \theta \csc \theta + \csc^2 \theta \]
Since $1 + \tan^2 \theta = \sec^2 \theta$:
\[ 2 \tan \theta + 1 + 2 \cot \theta + \cot^2 \theta = 2 \sec \theta \csc \theta + \csc^2 \theta \]
And $1 + \cot^2 \theta = \csc^2 \theta$ so:
\[ 2 \tan \theta + 2 \cot \theta = 2 \sec \theta \csc \theta \]

Multiply by $\sin \theta \cos \theta$:
\[ 2 \sin^2 \theta + 2 \cos^2 \theta = 2 \]
which is correct.

\subsection*{chords of a circle}
In our work on arcs and chords of a circle, we found that equal angles on the perimeter of the circle subtend equal arcs.  So, for example, in this figure the angle $s$ subtends the same arc in both panels, and a chord of the same length, regardless of where it intersects the perimeter.

\begin{center} \includegraphics [scale=0.4] {circle_chord.png} \end{center}

We also showed that, as in the right panel, the angle $APB$ is a right angle, since $AB$ is a diameter of the circle.  

Therefore, $PB$ is the opposite side in a right triangle.  As a result:

\[ \frac{PB}{2r} = \sin s \]
and this is true whether or not one of the sides flanking $s$ is a diameter of the circle.

A second proof of this is the following:  erect the perpendicular bisector of $PB$.
\begin{center} \includegraphics [scale=0.4] {circle_chord_2.png} \end{center}

The small triangle containing the dotted line as one edge and the vertex at $B$ is similar to the large triangle $ABP$, because the angle at the vertex $B$ is shared and the two sides $PB$ and $AB$ are in proportion.

Therefore, the small triangle is a right triangle, the central angle is $s$, and now one-half of $PB$ divided by $r$ is the sine of $s$.
\[ \sin s = \frac{PB/2}{r} = \frac{PB}{2r} \]

$\square$

\end{document}