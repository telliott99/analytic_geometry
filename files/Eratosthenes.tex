\documentclass[11pt, oneside]{article} 
\usepackage{geometry}
\geometry{letterpaper} 
\usepackage{graphicx}
	
\usepackage{amssymb}
\usepackage{amsmath}
\usepackage{parskip}
\usepackage{color}
\usepackage{hyperref}

\graphicspath{{/Users/telliott/Dropbox/Github-Math/figures/}}
% \begin{center} \includegraphics [scale=0.4] {gauss3.png} \end{center}

\title{Circular orbits}
\date{}

\begin{document}
\maketitle
\Large

%[my-super-duper-separator]

This part of the book is focused on geometry, and we take a look at Eratosthenes in this chapter as an important Greek scholar.

The widely held theory, that the ancient world believed the earth to be flat, is just wrong.  People with any level of sophistication not only knew the earth is roughly spherical but also knew its size within a few percent of the true value.

One likely basis is the false story that Columbus had trouble getting financing for his proposed trip to China because everyone thought he would fall off the edge of the earth.  This was a tall tale invented by Washington Irving, who also made up several remarkable fables about George Washington.

The real reason the Italians and the Portuguese thought Columbus would fail is that they had a pretty good idea of the size of the spherical earth and thus of the distance to China, while the over-optimistic Columbus believed it was about half the true value.  The prospective financiers knew that he was not able to carry the supplies necessary for a trip of this length.

Morris Kline (\emph{Mathematics and the Physical World}) says that the error is due to geographers after Eratosthenes, who reduced the estimated circumference from 24,000 to 17,000 miles.

\subsection*{Eratosthenes}

Views of the Greek philosophers on the earth and its sphericity are detailed here

\url{https://www.iep.utm.edu/thales/#SH8d}

Here is a partial quotation:

\begin{quote}
There are several good reasons to accept that Thales envisaged the earth as spherical. Aristotle used these arguments to support his own view [...] . First is the fact that during a solar eclipse, the shadow caused by the interposition of the earth between the sun and the moon is always convex; therefore the earth must be spherical. In other words, if the earth were a flat disk, the shadow cast during an eclipse would be elliptical. Second, Thales, who is acknowledged as an observer of the heavens, would have observed that stars which are visible in a certain locality may not be visible further to the north or south, a phenomen[on] which could be explained within the understanding of a spherical earth.
\end{quote}

\url{https://en.wikipedia.org/wiki/Eratosthenes}

Eratosthenes (ca. 276 - 195 BCE) measured the circumference of the earth from this observation:  at high noon on June 21st there was no shadow seen at Syene, allegedly from a stick placed vertically in the ground.  Some people say a deep well had the bottom illuminated at midday.

Syene is presently known as Aswan.  It is on the Nile about 150 miles upstream of Luxor, which includes the famous site called the Valley of the Kings.  At 24.1 degrees north latitude, Aswan or Syene is close enough to having the sun directly overhead on June 21.  (The "Tropic of Cancer" is at 23 degrees, 26 minutes north).

\begin{center} \includegraphics [scale=0.6] {aswan.png} \end{center}

Alexandria was a famous center of learning of the ancient world, and Eratosthenes was hired by the pharaoh Ptolemy III to be the librarian in 245 BCE.  Alexandria lies on the Mediterranean some 500 miles north of Syene, and anyone there who was looking could observe that at high noon on June 21st there \emph{was a shadow}.  This shadow Eratosthenes measured to be some 7 degrees and a bit (7 degrees and 10 minutes).

\begin{center} \includegraphics [scale=0.4] {eratosthenes.png} \end{center}

A full 360 degrees divided by 7 degrees and a bit is approximately 50.  So we can calculate on this basis that the circumference of the earth is about $50 \times 500 = 25000$ miles.  That's pretty close to the correct value.

For this calculation, we assume that the sun's rays are effectively parallel (not a bad assumption given a distance of 93 million miles).  Then we just use this:

\begin{center} \includegraphics [scale=0.3] {eratosthenes2.png} \end{center} 

an application of the alternate-interior-angles theorem.

It is curious how the distance from Alexandria to Syene was calculated. 

Kline:

\begin{quote} Camel trains, which usually traveled 100 stadia a day, took 50 days to reach Syene.  Hence the distance was 5000 stadia...It is believed that a stadium was 157 meters.\end{quote}

We obtain
\[ 157 \times 5000 \times 50 = 39,250 \ \text{km} \]
That's a much better estimate than a method that relies on camels really deserves.

\subsection*{Al-Biruni}

I found another method for measuring the size of the earth in Acheson's geometry book.

\begin{center} \includegraphics [scale=0.5] {al_biruni.png} \end{center} 

In the figure, the circle is the earth, of radius $R$, $h$ is the height of a convenient mountain, and $D$ is the distance to the horizon, which is tangent to the earth's radius.

Recall from the tangent-secant theorem 
\[ D^2 = h(2R + h) \]

We neglect $h^2$ compared to the other term so
\[ D^2 \approx 2Rh \]

About 1019 C.E., Al-Biruni obtained a value for $R$ equivalent to 3939 miles.

Note:  I have to look into why Acheson says this.  The standard treatment of the Al-Biruni's method uses the distance to the horizon plus the angle between a ray to the horizon and the horizontal or vertical axis at the point on the mountain.

\subsection*{Aristarchus}

Aristarchus of Samos (310-230 BCE) wrote a famous book in which he calculated the relative sizes of the sun and the moon and their distances from earth.

One straightforward observation is that the apparent size of the sun and moon in the sky is about the same.  This can be seen during a solar eclipse, or observed at any other time by holding a disk up at a fixed distance from the eye, (while taking care to block most of the sun's rays).  The value is approximately one-half degree.

Since the distance to the sun is much greater than that to the moon (see below), we can infer that the sun is much larger than the moon.

The central idea of Aristarchus is that, at half moon, the geometry of the three orbs is like this:

\begin{center} \includegraphics [scale=0.4] {half_moon.png} \end{center}

In other words, when the phase is half moon and that moon is exactly overhead, the sun has not yet set, but is a bit above the horizon. 

If $S$ is the distance to the sun and $L$ is that to the moon, he estimated that

\[ 18 < \frac{S}{L} < 20 \]

with the same ratio for their sizes.  Unfortunately, this is not a particularly good estimate.  The true value is about 390.  Aristarchus obtained a value of 20 for the Earth-Moon distance in Earth radii.  The correct value is about 60.  Much better estimates were obtained later, by Hipparchus and Ptolemy.

However, Aristarchus made up for this by being the first person to propose a heliocentric theory of the solar system:  that the earth and planets rotate around the sun.

\url{https://en.wikipedia.org/wiki/On_the_Sizes_and_Distances_(Aristarchus)}

\subsection*{quick estimate}

Here is an estimate for the earth-moon distance based on a lunar eclipse.

One measures the time it takes for a complete, total eclipse.  From the first shadow of the earth on the moon to the last, that time is about 3 hr.  The moon has moved approximately 1 earth diameter in its orbit in that time.

However, we must correct for the fact that the first and last shadows occur on opposite edges of the moon.  It was noted that the shape of the eclipse suggests the earth's diameter (at that distance) is about 2.5 moon diameters.  So the moon has actually moved (2.5 + 1.0)/2.5 = 1.4 earth diameters in the given time.  The relevant time becomes 2.14 hr.

Any correction for the true size of the earth's diameter is minimal because the earth-moon system is so far from the source of illumination.

The other piece of information we need is the time for a full revolution, one lunar cycle.  This part is tricky.  Naively, you'd look for the moon to be in the same place against the fixed stars (27 days, c. 8 hr).  This is off because the earth has moved in the meantime --- there is a parallax error.  As a rough correction, mutliply by 360/330 degrees.  The result in hours is 715.

The circumference of the orbit is then

\[ 715 / 2.143 = 333 \]
earth diameters.

This gives a radius of 53 earth diameters, which is not too far from 60.


\end{document}